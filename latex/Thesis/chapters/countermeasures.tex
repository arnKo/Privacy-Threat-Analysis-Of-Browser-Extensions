% !TeX spellcheck = en_US

\chapter{Countermeasures}
\label{chp:countermeasures}

	In this chapter, we present approaches to counter malicious extension that also affect our implemented system.
	
	We designed our system to be able to bypass some detection mechanisms. For that purpose, we first identify the current user and only if it was successful we fetch the source code for explicit attacks. This allows us to bypass a static analysis because malicious code is only present at runtime. Furthermore, we presented the dynamic analysis tools \textit{Hulk} and \textit{WebEval} in \autoref{sec:relatedWorks:detectionOfMaliciousExtensions}. These analyze extensions in a sandboxed environment. If we are able to successfully identify these environments, we can bypass the detection. However, if the analysis takes place while the browser is used by an end user, our malicious code will be detected. 
	
	Some previously proposed approaches modify the browser's JavaScript engine to allow the detection of malicious scripts by tracking the flow of information \cite{Dhawan:2009:AIF:1723192.1723250, ndss2007xss} or auditing called functions \cite{Hallaraker:2005:DMJ:1078029.1078861, cs2015sentinel}. The proposed systems to track the flow of information label data that may contain sensitive information and alert the user if these data is about to be transferred to an untrusted source, such as a remote server. These would certainly detect the malicious behavior of our system because we rely on collecting user data and transferring it to a remote server. The proposed auditing systems focus on the deprecated extension architecture of Firefox's Add-ons and remotely fetched scripts in web pages and are therefore not directly applicable to our work. However, it is conceivable to implement an auditing system for cross-browser extensions. The API provided by the browser and the separation between content scripts and the natively stored DOM provide the possibilities to monitor all calls to browser internal functionality and interactions with a web page. Such a monitoring system in combination with a thoughtful evaluation would certainly detect our malicious behavior. The monitoring of browser API calls would detect our implemented attacks and the monitoring of the DOM would detect our collection of user data.

	We presented another approach to increase the privacy of a user by default in \autoref{sec:relatedWorks:threatAnalysis}. Liu et al. proposed an improved permission system that can prevent some of our implemented attack vectors. If developers request only necessary privileges for their extensions, the amount of components that we can integrate into their implementations would decrease. The proposed restriction on origins that an extension is allowed to add to a web page can prevent our implementations that execute web requests using DOM elements, and the restriction of DOM elements that contain sensitive information can prevent our components that collect user data. However, many extensions still need privileges for their legal purpose that our components need, too. This is a general problem that most functionality can be used in a benign or malicious way. Therefore, a permission system can not protect a user from a malicious extension because the permissions alone do not provide an accurate statement whether or not an extension is indeed malicious. 
