% !TeX spellcheck = en_US

\chapter{Conclusion}
\label{chp:conclusion}

	In this work, we presented an extension architecture supported by most moder browsers. We discussed the research that directly affected the architecture, and gave an overview of its general structure and the security mechanisms that should protect the user if an extension is compromised. We presented previously conducted analyses of possible privacy threats that emerge from the use of browser extensions. The researchers demonstrated the possibility to execute attack vectors in browser extensions. We decided to prove that it is even possible to execute targeted attacks against identified users. \\
	We analyzed the threats that arise from the extension's capabilities, and revealed that the unrestricted access to the web page's DOM poses the biggest threat to the user's privacy. Afterwards, we presented our proof-of-concept which is capable of integrating explicit attack vectors into an existing extension based on the extension's already available privileges. Our system executes targeted attacks by identifying the current user and fetching the source code for an attack vector only if the identification was successful. We evaluated the applicability of our implementation by analyzing the requested privileges of popular extension, and demonstrated that we are indeed able to integrate our attack vectors into most of the analyzed extensions. \\
	Finally, we presented existing countermeasures against malicious extensions that can affect our implementation.
	
	With our work, we proved that many privileges, that an extension can request, pose a threat to user and his privacy. Especially, the combination between a user identification an remotely loaded scripts allows to execute targeted attacks while bypassing some detection mechanism. The malicious code is only fetched at runtime after a successful identification. Therefore, it is not available to a static analysis. Furthermore, if we detect that the extension is currently targeted by a dynamic analysis, we can fetch a harmless script instead of an attack. The security features of cross-browser extensions mitigate some threats by default. Nevertheless, our extension analysis revealed that many extensions override the default privileges which facilitates the integration of malicious behavior.	All in all can be said, that the user should be cautious when installing an extension and take an extra look at the extension's requested privileges.