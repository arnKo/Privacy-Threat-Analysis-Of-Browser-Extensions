% !TeX spellcheck = en_US

\chapter{Introduction}

	Web browsers are a common tool to interact with the Internet. They allow users to view web content, interact with web applications, and communicate with other people around the world. To enhance this interaction, browsers support extensions that programmatically can change the look and behavior of web pages and add further functionality to the browser. For that purpose, the browser provides additional interfaces to its extensions that provide access to browser-internal features such as the user's bookmarks, stored cookies, or graphical user interfaces.
	
	The wide range of functionality that a browser provides to its extensions is used by criminals to attack the extension's user. For example, the extensions \textit{Febipos} and \textit{Kilim} which were discovered in 2013 target the user's social media accounts and spread spamming content without the user's consent \cite{febipos, kilim}. Both are known to security vendors and modern anti-virus software is configured to detect and remove them. \\
	However, even popular and at first glance harmless extensions may execute behavior that harms the user and his privacy. For example, the popular extension \textit{Hola}\footnote{\url{https://chrome.google.com/webstore/detail/gkojfkhlekighikafcpjkiklfbnlmeio} [accessed 2016-08-30]} which currently has about 9 million users received negative publicity after its botnet-like behavior was uncovered. The extension's base functionality allows to bypass regional restrictions on web content by routing the user's requests to other users of their network to whom the restrictions do not apply. This creates a VPN-like network that masks the origin of the user's requests. The difference to a common VPN is that each defaulting customer acts as an endpoint for the network. Therefore, the user is regularly exposed to the open Internet by traffic from other users that took on his IP address. This creates a threat for the user because malicious requests, such as to illegal content or to execute web attacks, originate from his device. Furthermore, the company responsible for the extension provides a second service that allows paying customers to route targeted traffic through the network. This second service was not revealed to the extension's users until the operator of a web platform claimed that the network was used to execute DDoS attacks against his servers \cite{holaFaqRewrite, 8chanHola}. 
	
	The problem of malicious browser extensions is well known to researchers and industry experts. They try to control the misuse of extensions by analyzing possible threats \cite{Liu12chromeextensions:, liu2011botnet, TerLouw:2007:EWB:1420581.1420583}, improving the architecture \cite{Barth10protectingbrowsers, Carlini:2012:EGC:2362793.2362800, cs2015sentinel, TerLouw:2007:EWB:1420581.1420583}, and developing systems to detect malicious extensions \cite{184485, 190984, Bandhakavi:2011:VBE:1995376.1995398}. For exmaple, Google Chrome developers analyzed 99,818 Chrome extension published to the official distribution platform between 2012 and 2015. They identified 9,523 malicious extensions which were nearly 10\% of the analyzed extensions. As the number of Chrome users increases \cite{w3browserStats, statcounter, netmarketshare}, we assume the the number of victims of malicious extensions increases, too.

	In our work, we demonstrate an attack pattern that holds a big threat to particular users, and at the same time is able to bypass some detection techniques. We show that an extension is able to identify its current user and purposefully attack him. Additionally, loading the attack's source code only after a successful identification allows to bypass some detection mechanisms, based on a code analysis, since the malicious implementation is fetched only for a specific user. In order to prove the threat that this pattern poses, we have analyzed an extension's capabilities and designed a system to integrate malicious implementations based on the extension's privileges. We divided our system between the identification of the current user, covert communication with a remote server, and explicit attacks. For each part, we implemented interchangeable components which use different functionality. Afterwards, we analyzed existing extensions whether we are able to integrate our implemented components. We discovered that many existing extensions are very complex and request many privileges which facilitates the integration of our malicious components and increases the threat for the user.
	
	Although possible attacks were previously analyzed in related works, we checked the feasibility of our implemented attack vectors in most modern browsers, which are Google's \textit{Chrome} browser, Mozilla's \textit{Firefox} browser, Microsoft's \textit{Edge} browser, and the \textit{Opera} browser. \autoref{tab:browserShare} shows browser usage statistics from different companies  that provide analytic frameworks for websites \cite{w3browserStats, statcounter, netmarketshare}. Therefore, they are able to collect user information from many different web pages and generate global statistics. A general and exact statistic about browser usage is currently not possible, because the browsers' companies do not publish the numbers of active users or are not able to determinate them. The presented statistics show that our work is currently applicable to extensions of around 70\% of browser users.

	\begin{table}
		\centering
		\begin{tabular}{|l|l|l|l|}
			\hline
			\textbf{Browser} & \textbf{w3counter} & \textbf{StatCounter} & \textbf{NetMarketShare} \\ \hline
			Chrome & 59.5\% & 62.38\% & 50.95\% \\ \hline
			Internet Explorer & 8.6\% & 10.73\% & 29.60\% \\ \hline
			Firefox & 10.1\% & 15.43\% & 8.12\% \\ \hline
			Safari & 13.1\% & 4.59\% & 4.51\% \\  \hline
			Opera & 2.6\% & - & 1.40\% \\ \hline
			Edge & 1.6\% & 3.04\% & 5.09\% \\ \hline
		\end{tabular}
		\caption{Different statistics about the global browser share of July 2016 \cite{w3browserStats, statcounter, netmarketshare}}
		\label{tab:browserShare}
	\end{table}	
	
\section{Goals}
	
	To goals and objectives of our work are as follows.
	\begin{itemize}[nosep]
		\item Provide further information to our topic including related works, used terminology, and an overview of the extension architecture supported by most modern browsers.
		\item Analyze the focused extension architecture according to potential threats to the user's privacy.
		\item Demonstrate discovered threats by designing and implementing a system which integrates several attack vectors against end users, depending on provided functionality.
		\item Evaluate the applicability of our system by analyzing whether we are able to integrate the system's components into existing extensions. 
		\item Discuss existing countermeasures. 
	\end{itemize}

\section{Outline}
	
	The remainder of this paper is structured as follows. In \autoref{chp:relatedWorks} we review related works to our topic which include the origin of our focused extension architecture (\autoref{sec:relatedWorks:extensionArchitecture}), threat analysis of malicious browser extensions (\autoref{sec:relatedWorks:threatAnalysis}), and the detection of malicious extensions (\autoref{sec:relatedWorks:detectionOfMaliciousExtensions}). \\
	We provide additional background information in \autoref{chp:background} including less known terminology that we use in our paper in \autoref{sec:terminology} and an overview of the extension architecture that we focus in \autoref{sec:extensionArchitecture}. \\
	In \autoref{chp:threatAnalysis}, we present the results of our theoretical threat analysis. \autoref{sec:threatAnalysis:contentScripts} contains threats that arise from the extension's access to a displayed web page. \autoref{sec:threatAnalysis:api} contains threats that arise from the functionality provided by the browser to its extensions. \\
	Afterwards, we present our design that acts as a proof-of-concept for our theoretical analysis in \autoref{chp:design}. We present our implementations to identify the extension's current user in \autoref{sec:identification}, our implementations to converted communicate with a remote server in \autoref{sec:communication}, and our implementations of explicit attack vectors in \autoref{sec:execution}. \\
	We tested the applicability of our implementations against real-world extensions and present the results of this analysis in \autoref{chp:extensionAnalysis}.
	Then, we present existing and proposed countermeasures against malicious extensions in \autoref{chp:countermeasures}.
	Finally, we discuss further work in \autoref{chp:furhterWork} and give a brief conclusion to our work in \autoref{chp:conclusion}.
	