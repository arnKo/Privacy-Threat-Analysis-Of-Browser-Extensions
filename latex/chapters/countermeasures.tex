% !TeX spellcheck = en_US

\chapter{Countermeasures}
\label{chp:countermeasures}

	In this chapter, we present two approaches to counter malicious extension. First, we briefly present existing analysis tools to detect malicious code in extension. And then, we describe proposed improvements of the cross-extension's permission system which restricts the extension's privileges by default even more and therefore hampers the integration of our malicious components.

\section{Detect Malicious Behavior}
\label{sec:detectMaliciousBehavior}
	 
	Mozilla's security experts review Add-ons that they distribute on the official web store to reject malicious implementations. This could also be done for the cross-browser extensions. Indeed, a research team already developed a successful system called \textit{WebEval} to identify malicious extensions. For three yeas, they collected each version of all extensions published to the Chrome Web Store and fed them into their system. They detected about 9,000 malicious extensions which were nearly 10\% of all analyzed extensions.
	
	Furthermore, WebEval uses a machine learning algorithm to evaluate each extension and potential malicious findings are verified by human experts. The researchers measured the overall accuracy of their system which increased over time although they experienced drops in the accuracy when new threats emerged. \\
	A disadvantage of WebEval is that it executes the extensions in a sandboxed environment. Therefore, not all malicious interactions with web content can be monitored. For example, the system has no account creation on-the-fly and is therefore not able to analyze the extension's behavior on account-restricted web pages.
	
	Other researchers developed systems to monitor the flow of information in JavaScript-base applications such as extensions \cite{Dhawan:2009:AIF:1723192.1723250, Hallaraker:2005:DMJ:1078029.1078861, cs2015sentinel,ndss2007xss}. They blocked the flow if sensitive information are transfered to untrusted targets such as remote servers. However, all approaches modify the browser's JavaScript interpreter to enable the tracking and produce a not irrelevant overhead.
	
	We designed our system to be difficult to detect by an analysis. For that purpose, we first identify the current user and only if it was successful we fetch the source code for explicit attacks. This allows us to bypass detection tools that work in sandboxed environments. If the browser itself would analyze extension while the user navigates through the world wide web and detect malicious behavior, we were not able to execute attacks.

\section{Improved Permissions}
\label{sec:improvedPermissions}

	Another strategy to increase the privacy of a user was proposed by Liu et al. \cite{Liu12chromeextensions:}. They conducted a threat analysis of Chrome extensions and discovered that the extension's unrestricted access to the web page and the possibility to execute unrestricted web requests posse the most harm to the user's privacy. As we showed ourselves, a malicious extension is able to extract and transfer sensitive user information from any web page that the user visits using only a single content script. The researchers stated that the current permission system does not comply to the principle of least privileges and proposed a modified permission system to mitigate the found threats. 
	
	The extension's components namely the background and content scripts share the same set of permissions declared in the manifest. This leads to some components having permissions they do not need and therefore not being least-privileged. If the extension requests host permissions for a particular origin, the extension's background is allowed to inject scripts in matching web pages and both - the background and content scripts - are allowed to execute web requests to the origin. Liu et al. proposed separated permission sets for each component and to split host permissions based on particular operations such as to inject a script or execute a web request. However, a content script is still able to execute unrestricted web requests by adding an element to the DOM that fetches a resource from a remote origin. We also use this strategy for our design to execute a request to any arbitrary remote server. To mitigate this threat, the researchers proposed an additional permission that restricts a content script and allows that only listed origins can be added as sources to the DOM.
	
	An extension's content script has unrestricted access to the web page and its content. This allows a malicious extension to extract sensitive information. Liu et al. proposed a categorization of DOM elements to identify container of sensitive information and restricted access to these elements by default. \textit{High level} elements such as input elements of type password or hidden contain inherently sensitive information. \textit{Medium level} elements may contain sensitive information. To identify these elements, the researchers proposed a catalog with regular expressions that match code words such as \textit{username}. Finally, every element else is categorized as \textit{low level}. By default, an extension should have only access to low level elements and the extension may request access to the other levels by a proper permission.
	
	The proposed improvements of the permission systems would indeed hamper our design. If developers request only necessary permissions for their extensions, the amount of components that we can integrate into their implementation would decrease. But still, many extensions need permissions for their legal purpose that also our components need. This is a general problem that most functionality can be used in a benign or malicious way. Therefore, a permission system can not protect a user from a malicious extension because the permissions alone do not provide an accurate statement whether or not an extension is indeed malicious. 