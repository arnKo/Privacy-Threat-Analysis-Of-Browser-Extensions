% !TeX spellcheck = en_US

\chapter{Threat Analysis}
\label{chp:threatAnalysis}

	An extension can use a wide range of different features to enhance the user's interaction with a web page. We want to show that these features may be used by a developer of a malicious extension to harm the user. For that purpose, we have analyzed the extension's capabilities and found potential threats. We have found several permissions and modules that an attacker may use to harm the user's privacy, use his device to launch attacks against others, or remove privacy preserving measures and therefore support attacks from malicious web pages.  

\section{Content Scripts}
\label{sec:threatAnalysis:contentScripts}

	A big threat to the user's privacy that an extension posses is its full access to a web page. If the extension uses a content script with a URL pattern that matches any web page, it has access to any user data that the page contains. There exists no further restriction such as additional permissions to access password fields or other container of sensitive data. Furthermore, an extension is able to execute cross-origin requests to any arbitrary web page. For that purpose, it can use the mechanics of a DOM element that fetches a resource from a remote server such as an iframe, image, script, or link. We have listed some potential attack scenarios:
	
	\begin{itemize}
		\item \textbf{Steal User Data From Forms} Any information the user transmits over a form in a web page is accessible for an extension. To steal this information, the extension adds an event listener which is dispatched when the user submits the form. At this point in time, the extension can read out all information that the user has entered in the form. This approach gives the attacker access to the user's personal information such as his address, email, phone number, or credit card number but also to identification data such as social security number, identity number, or credentials. Especially username and password for a website's login are typically transmitted with a form.
		
		\item \textbf{Steal Displayed User Data} Any information about the user that a web page contains is accessible for an extension. To steal this information, the attacker has to explicit know where it is stored in the web page. This is mostly a trivial task, because most web pages are public and the attacker is therefore able to analyze the targeted web page's structure. With this attack, an attacker is able to obtain a broad range of different information such as the user's financial status from his banking portal, his emails and contacts, his friends and messages from social media, or bought items and shopping preferences.
		
		\item \textbf{Modify Forms} An extension can add new input elements to a form. This tricks the user into filling out additional information that are not necessary for the website but targeted by the attacker. For that purpose, the extension adds the additional input fields to the form when the web page loads, steals the information when the user submits the form, and removes the additional fields afterwards. The last step is necessary because the web application would return an error the form's structure was modified. This attack will succeed if the user does not know the form's structure beforehand. To decrease the probability that the user knows the form already, the extension can determinate whether or not the user has visited the web page to an earlier date before executing the attack. 
		
		\item \textbf{Modify Links} An extension can modify the URL of a link element to redirect the user to a another web page. This page may be malicious and infect the user's device with malware or it may be a duplicate of the web page to which the link originally led and steal the user's data. But the attacker may also enrich himself through the extension's users. Some companies pay money for every time someone loads a specific web page. The attacker can redirect the user to this web page and gain more profit.

		\item \textbf{Denial Of Service} If the source attribute of a DOM element such as an iframe or image changes, the browser sends a HTTP request to fetch the desired resource from the targeted server. An extension can add an unlimited number of elements to the web page's DOM and thereby flood a targeted server with requests. The attack is even more potent if the malicious extension is installed on many browsers and each executes the attack simultaneously.
	\end{itemize}
	
\section{Browser API}
\label{sec:threatAnalysis:api}
	
	In this section, we present found threats that arise from API modules and are unlocked by a declared permission. Most of the permissions give access to an API module whose features may be misused in some form. 
	
	A developer may declare permissions as optional if they are not necessary for his extension's basic functionality. At runtime, the user is prompted to grant the extension the access to a requested permission. However, the user can not deny a once granted permission by himself. Therefore, a optional permission is as dangerous as standard permission in terms of potential threats because once it was accepted by the user it stays active and the extension may misuse the provided functionality. Furthermore, empirical studies showed that users tend to ignore security warnings which refers also to security relevant prompts \cite{180371, sunshine2009crying}.
		
	The following paragraphs show the threats which we found associated with a permission. Each paragraph has the permissions's name as heading which is equal to a probably associated module. Additionally, if the permission results in a warning on the extension's installation, we added it to the paragraph.
	
\newenvironment{permissionwarning}{%
	\setlength\topsep{4pt}
	\setlength\parskip{0pt}
	\itshape
\begin{center}
	}{%
\end{center}
}

\paragraph{activeTab}
	The active tab permission grants an extension temporary access to the currently active web page if the user invokes the extension. 
	
	An extension may execute any malicious behavior that does not target a specific web page without the need to declare a content script or host permissions. This allows to decrease suspicious permissions such as the access to all web pages because the \texttt{activeTab} permission does not result in a warning on installation.

\paragraph{background}
	This permission is not related to an API module. Instead, if one or more extensions with the background permission are installed and active, the browser starts its execution with the user's login into the operating system without being invoked and without opening a visible window. The browser will not terminate when the user closes its last visible window but keeps staying active in the background. This behavior is only implemented in Chrome and can be disabled generally in Chrome's settings.
	
	A malicious extension with this permission can still execute attacks even when no browser window is open.
	
\paragraph{bookmarks} 
	This module gives access to the browser's bookmark system. The extension can create new bookmarks, edit existing ones, or remove them. It can also search for particular bookmarks based on parts of the bookmark's title, or URL and retrieve the recently added bookmarks.
	
	The user's bookmarks give information about his preferences and used web pages. This may be used to identify the currently active user or to determinate potential web page targets for further attacks.
	
	On installation, an extension with this permission shows the user the following warning:
	\begin{permissionwarning}
		Read and modify your bookmarks
	\end{permissionwarning}
	
\paragraph{contentSettings} 
	The browser provides a set of \textit{content settings} that control whether web pages can include and use features such as cookies, JavaScript, or plugins. This module  allows an extension to overwrite these settings on a per-site basis instead of globally.
	
	A malicious extension can disable settings which the user has explicitly set. This will probably decrease the user's security while browsing the web and support malicious web pages.
	
	On installation, an extension with this permission shows the user the following warning:
	\begin{permissionwarning}
		Manipulate settings that specify whether websites can use features such\\as cookies, JavaScript, plugins, geolocation, microphone, camera etc.
	\end{permissionwarning}
	
\paragraph{cookies} 
	This module give an extension read and write access to all currently stored cookies, even to \textit{httpOnly} cookies that are normally not accessible by client-side JavaScript.
	
	An attacker may use an extension to steal session and authentication data which are commonly stored in cookies. This allows him to act with the user's privileges on affected websites. Furthermore, an malicious extension may restore deleted tracking cookies and thereby support user tracking attempts from websites.
	
\paragraph{downloads} 
	This module allows an extension to initiate and monitor downloads. Some of the module's functions are further restricted by additional permissions. To open a downloaded file, the extension needs the \texttt{downloads.open} permission and to enabled or disable the browser's download shelf, the extension needs the permission \texttt{downloads.shelf}.  
	
	With the additional permission \texttt{downloads.open}, a malicious extension can download a harmful file and execute it. Another malicious approach is to exchange a benign downloaded file with a harmful one without the user noticing. 
	
\paragraph{geolocation}
	The HTML5 geolocation API provides information about the user's geographical location to JavaScript. With the default browser settings, the user is prompted to confirm if a web page want's to access his location. If an extension uses the geolocation permission, it can use the API without prompting the user to confirm.
	
	On installation, an extension with this permission shows the user the following warning:
	\begin{permissionwarning}
		Detect your physical location 
	\end{permissionwarning}
	
\paragraph{management}
	This module provides information about currently installed extensions. Additionally, it allows to disable and uninstall extensions. To prevent abuse, the user is prompted to confirm if an extension wants to uninstall another extension. 
	
	An attacker may use the feature to disable another extension to silently disable security relevant extension such as 
	\textit{Adblock}\footnote{\url{https://chrome.google.com/webstore/detail/gighmmpiobklfepjocnamgkkbiglidom}}, 
	\textit{Avira Browser Safety}\footnote{\url{https://chrome.google.com/webstore/detail/flliilndjeohchalpbbcdekjklbdgfkk}}, or 
	\textit{Avast Online Security}\footnote{\url{https://chrome.google.com/webstore/detail/gomekmidlodglbbmalcneegieacbdmki}}.
	
	On installation, an extension with this permission shows the user the following warning:
	\begin{permissionwarning}
		Manage your apps, extensions, and themes 
	\end{permissionwarning}
	
\paragraph{proxy}
	Allows an extension to add and remove proxy server to the browser's settings. If a proxy is set, all requests are transmitted over the proxy server.
	
	This feature may be used by an attacker to send all web requests over a malicious server. For example, a server that logs all requests and therefore steal any use information that is transmitted unsecured.
	
	On installation, a extension with this permission shows the user the following warning:
	\begin{permissionwarning}
		Read and modify all your data on all websites you visit 
	\end{permissionwarning}
	
\paragraph{system}
	The \texttt{system.cpu}, \texttt{system.memory}, and \texttt{system.storage} permissions provide technical information about the user's machine.
	
	These information may be used to create a profile of the current user's machine and identify him on later occasions.
	
\paragraph{tabs}
	An extension can access the browser's tab system with the tabs module. This enables the extension to create, update, or close tabs. Furthermore, it provides the functionality to programmatically inject content scripts into web pages and to interact with a content script which is active in a particular tab. To inject a content script, the extension needs a proper host permission that matches the tab's current web page. The tabs permission does not restrict the access to the tabs module but only the access to the URL and title of a tab. 
	
	A malicious extension may prevent the user from uninstalling it by closing the browser's extensions tab as soon as the user opens it. The programmatically injection takes a content script either as a file in the extension's bundle or as a string of code. Therefore, a malicious extension may inject remotely loaded code into a web page as a content script that executes further attacks.
	
	On installation, an extension with this permission shows the user the following warning:
	\begin{permissionwarning}
		Access your browsing activity 
	\end{permissionwarning}		
	
\paragraph{webRequest}

	This module enables an extension to modify outgoing web requests and their responses. For that purpose, it provides several events which are fired at different stages of a web request's life cycle. To get access to the web requests, the extension needs the \texttt{webRequests} permission and proper host permissions that match the request's URL. Additionally, the \texttt{webRequestBlocking} permission is needed in order that the extension can block the web request's processing and manipulate it. Blocking can be used to cancel or redirect the request and to modify the request's and the response's headers. 
	
	A malicious extension can use this module to remove security relevant headers such as a CSP , intercept outgoing requests, or redirect requests from benign to malicious web pages.
	
	This permission itself does not result in a warning when an extension that requires it is installed. But, to get access to the data of a web request the extension needs proper host permissions and these result in a warning. The often used host permissions \texttt{http://*/*}, \texttt{https://*/*}, and \texttt{<all\_urls>} result in the following warning:
	\begin{permissionwarning}
		Read and modify your data on all websites you visit 
	\end{permissionwarning}
	
