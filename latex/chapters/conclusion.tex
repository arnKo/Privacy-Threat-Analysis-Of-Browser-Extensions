% !TeX spellcheck = en_US

\chapter{Conclusion}
\label{chp:conclusion}

	We provided an overview of the cross-browser extension architecture. We outlined its general structure and the separation between the extension's background and content scripts and described the security features that should protect the user if an extension is compromised. Afterwards, we presented the results of our threat analysis which revealed that many modules of the browser API contain threats to the user and his privacy and that the biggest threat arises from the extension's full access to displayed web pages.\\
	As a proof-of-concept, we designed a system to integrate malicious behavior into existing extensions. Our design follows a pattern that allows to execute targeted attacks against identified users. First, we collect user information. Then, we transfer collected data to a remote server that is in charge of the identification. Finally, we load the source code for an attack after a successful identification and execute it. We implemented interchangeable components for each step that need different privileges. This structure allows us to integrate our components into a an extension based on the extension's already available privileges. Therefore, we do not change the extension's current privileges and the user will not be notified if the extension is updated. \\
	Finally, we demonstrated the applicability of our design to existing extensions. We conducted an analysis of the requested privileges of existing extensions which we collected from an official extension platform. Our results revealed that many extensions indeed possess privileges matching the needed privileges for our implemented components. Hence, the developers are able to integrate malicious behavior without changing requested privileges.
	
	With our work, we proved that many privileges an extension can request pose a threat to user. Especially, the combination between a user identification an remotely loaded scripts allows to execute targeted attacks whilst bypassing some detection mechanism because malicious code is not present most of the time. The security features of cross-browser extensions mitigate some threats by default. Nevertheless, our extension analysis revealed that many extensions override the default privileges which facilitates the integration of malicious behavior. All in all can be said, that the user should be cautious when installing an extension and take an extra look at the extension's requested privileges.