% !TeX spellcheck = en_US

\chapter{Conclusion}
\label{chp:conclusion}

	We provided an overview of the cross-browser extension architecture, outlined its general structure and the separation between the extension's background and content scripts, and described the security features that should protect the user if an extension is compromised. Then, we presented the results of our threat analysis which revealed that many modules of the browser API contain threats to the user and his privacy and that the biggest threat arises from the extension's full access to displayed web pages.\\
	As a proof-of-concept, we designed a system to integrate malicious behavior into existing extensions. Our design consists of three steps and each step consists of interchangeable components that need different privileges. This structure allows us to integrate our implemented components into a an extension based on the extension's already available privileges. Therefore, we do not change the extension's current privileges and the user will not be notified if the extension is updated. Furthermore, most of our system's code is not present on the extension's installation but instead is loaded remotely if the current user is successfully identified. This reduces the risk of detection because malicious code is only presen \\	
	Finally, we showed the applicability of our design to existing extensions. We analyzed the requested privileges of extensions with a high number of users which we fetched from an official extension platform. Our results revealed that many extension indeed possess privileges that match the needed privileges for our implemented attacks.
	
	With our work, we showed that extensions supported by the majority of modern browsers pose a threat to the user and his privacy and that existing, benign extensions currently possess the privileges to execute harmful behavior. \\	
	Our implementations proof that many privileges an extension can request may be used to violate the user's privacy or execute web attacks against him. Our design and extension analysis revealed that a benign extension is able to include harmful behavior without the user noticing. Especially remotely fetched scripts in combination with a user identification are difficult to detect, because the malicious code is not present most of the time. \\
	The security features of cross-browser extensions mitigate some threats by default. But, like our extension analysis revealed, many extensions override the default privileges which facilitates the integration of malicious behavior.