% !TeX spellcheck = en_US

\section{Terminology}
	% We explain therms we use in this paper

\subsection{Browser}
	% Application to acces the Internet, shows web pages, allows to interact with web applications, sends HTTP/HTTPS requests to servers and processes HTTP-responses

\subsection{Browser Extension}
	% Additional application that enhance browser, can use browser functions (API)

\subsection{Web Application}
	% Application executed remotely over the network and implements client-server architecture

\subsection{Web Page}
	% Document fetched from a remote server and displayed by the browser, consists of HTML as structure, CSS as design, and JavaScript to manipulate content dynamically,

\subsubsection{Document Object Model (DOM)}
	% Browser intern representation of a web page, standardized, accessible from JavaScript as document object, 

\subsubsection{Same Origin Policy (SOP)}
	% Security policy, JavaScript has no access to any document with another origin than the JavaScript, 

\subsubsection{Content Security Policy (CSP)}
	% Security policy, 

\subsubsection{XMLHttpRequest (XHR)}
	% Allows to make HTTP/HTTPS requests with JavaScript, used to dynamically load content or transfer information, 

\section{Extension Architecture}
	% we have analysed the architecture models of differend browsers, chrome extension/firefox add-ons/safari extensions, we show the general strucutre and differenses between the models,
	
\subsection{General Structure}
	% implemented with HTML/CSS/JS, two parts: extension core and content scripts, 

\subsubsection{Background} 
	% holds main logic, also includes additional UI such as popups or buttons in the browser toolbar, access to browser functionality,

\subsubsection{Browser API}
	% browser provides additional functions to its extensions, such as ..., 

\subsubsection{Content Scripts}
	% used to access the web page, has full access to the DOM, can manipulate and add content, generally injected with URL pattern, only injected into web page if URL matches, using wildecards to match multiple web pages at once, separated from background, can not directly call methods in background scripts, can exchange messages with background, only restricted access to browser API, how much is browser dependend, 
			
\subsection{Differences Between The Browsers}

	
\subsubsection{Chrome Extensions}
	% opera switch to chromium, firefox support for chrome extensions
	% model from barth et al, focused to protect the user from exploited extensions, secure extension model to prevent, three prinizples: privilege separation, least privileges, isolated world,
	
	\begin{itemize}
		\item \textbf{Privilege Separation}
			% separation different processes, only communicate over message channel, prevent leak of browser API functions to content scripts
		\item \textbf{Least Privileges} 
			% extension has by default no access to browser API, uses permissions to grant access, extension has to explicit state for what API modules it needs access, in case of exploit, attacker has no access to unrequested modules, smallers the attackers operating range,
		\item \textbf{Isolated World}
			% separation between content scripts among each other and the web page, each runs in own process with own JavaScript heap, not possible to call a function on another script, protection against a malicious web page altering a content script and attacking the extension, but they all share the web page's DOM, to prevent that a malicious web page overrides a DOM method and probably exploits a content script, each has own instance of document object that mirrows the browser intern DOM object, 
	\end{itemize}


\subsubsection{Firefox Add-ons}


\subsubsection{Safari Extensions}