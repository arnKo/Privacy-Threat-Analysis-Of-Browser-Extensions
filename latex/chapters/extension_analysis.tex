% !TeX spellcheck = en_US

\chapter{Extension Analysis}
	
	We want to show that our implementation is indeed applicable to real world extensions without the need to modify the extension's privileges. Therefore, we evaluated the privileges of popular extensions and analyzed which of our implemented components we can integrate into them. For that purpose, we compared the extension's declared permissions and content with the permissions and content needed for our components. 
	
\section{Preparations}

	To facilitate the analysis and in order that we do not have to list all components that we can integrate in the analyzed extension, we categorized our components and divided them into three groups:

	\begin{itemize}[nosep]
		\item[A] Content script or host permission for all web pages
		\item[B] Host permission for all web pages
		\item[C] API permission, combination of permissions, or modified CSP
	\end{itemize}
	
	Each group needs another set of privileges in order to work. Group A contains all components that use only a content script. If the extension hast host permissions to all web pages, we can programmatically inject the content script in any web page. Therefore, we can integrate a component that uses a content script in an extension that declares host permissions for all web pages. Group B contains all components that explicitly need host permissions. These often use an XHR or need access to the current web page but can not execute their behavior as a content script. Finally, group C contains all other components. Because it does not facilitate the analysis if we create a group for each component that uses a unique combination of privileges we decided to list these separately. \autoref{tab:componentCategorization} shows which component belongs to which group. 
	
	We fetched 15 extensions from the Google Chrome Web Store\footnote{\url{https://chrome.google.com/webstore/category/extensions}} for our analysis. Unfortunately, the web store does not provide the functionality to sort extension based on the amount of current users. Furthermore, the shown number of current users is cut if it is higher than 10,000,000. Therefore, we had to manually search through the web store and select extensions with as many users as possible to evaluate ourself. \autoref{tab:summaryEvaluatedExtensions} shows the extensions that we have fetched and analyzed sorted by the amount of users. %TODO maybe move to appendix
	
	\begin{table}[h]
		\centering
		\begin{tabular}{|l|l|r|} \hline
			\textbf{Extension} & \textbf{Version} & \textbf{Amount Users} (as of 2016-08-08) \\ \hline
			AdBlock & 3.1 &  10,000,000+ \\ \hline
			Adblock Plus & 1.12.1 & 10,000,000+  \\ \hline
			Adobe Acrobat & 15.1.0.1 & 10,000,000+ \\ \hline
			Avast Online Security & 11.1.0.955 & 10,000,000+ \\ \hline
			Avira Browser Safety & 1.11.0 & 10,000,000+ \\ \hline
			Grammarly for Chrome & 8.670.558 & 4,834,105~~~ \\ \hline
			Unlimited Free VPN - Hola & 1.15.403 & 8,547,308~~~ \\ \hline
			Google Hangouts & 2015.1204.418.1 & 5,639,654~~~ \\ \hline
			Google Translate & 2.0.6 & 5,554,424~~~ \\ \hline
			LastPass: Free Password Manager & 4.1.25 & 4,249,870~~~ \\ \hline
			Evernote Web Clipper & 6.9 & 4,056,700~~~ \\ \hline
			Google Mail Checker & 4.4.0 & 3,802,536~~~ \\ \hline
			Click\&Clean & 8.9.2 & 3,249,590~~~  \\ \hline
			IE Tab & 9.8.3.1 & 3,193,479~~~ \\ \hline
			Save to Pocket & 2.0.35 & 2,593,149~~~ \\ \hline
		\end{tabular}
		\caption{Summary of analyzed extension sorted by users.}
		\label{tab:summaryEvaluatedExtensions}
	\end{table}
	
	\begin{table}[h]
		\begin{tabular}{|l|l|l|p{0.36\textwidth}|} \hline
			\textbf{Group} & \textbf{Component} & \textbf{Section} & \textbf{Privileges (only for group C)} \\ \hline
			\multicolumn{4}{|l|}{\textbf{Identification}} \\ \hline
			C & Store identifier & \autoref{sec:storeIdentifier} & storage \\ \hline
			A & Web Beacon & \autoref{sec:webBeacon} &  \\ \hline
			A & History Sniffing & \autoref{sec:historySniffing} & \\ \hline
			C & History Sniffing & \autoref{sec:historySniffing} & history \\ \hline
			C & Bookmark Sniffing & \autoref{sec:bookmarkSniffing} & bookmarks \\ \hline
			A & Fingerprinting & \autoref{sec:additionalFingerprintingData} & \\ \hline
			A & Read Personal User Data & \autoref{sec:personalUserInformation} & \\ \hline
			
			\multicolumn{4}{|l|}{\textbf{Communication}} \\ \hline
			B & XHR & \autoref{sec:xhrCommunication} & \\ \hline
			A & IFrame & \autoref{sec:iframeCommunication} & \\ \hline
			C & XHR + eval & \autoref{sec:xhrFetching} & host permission + \texttt{unsafe\_eval} \\ \hline
			C & script element in background & \autoref{sec:scriptElementInBackground} & remote address in CSP \\ \hline
			A & script element in web page & \autoref{sec:scriptElementInContentScript} & \\ \hline
			
			\multicolumn{4}{|l|}{\textbf{Execution}} \\ \hline
			A & steal user data &  \autoref{sec:personalUserInformation}, \autoref{sec:stealFormData} & \\ \hline
			A & Manipulate requests & \autoref{sec:manipulateWebRequests} & \\ \hline
			C & Manipulate requests & \autoref{sec:manipulateWebRequests} & host permissions + webRequest + webRequestBlocking  \\ \hline
			A & Executed concealed attack & \autoref{sec:executeAttackInBackground} & \\ \hline
			A & DoS & \autoref{sec:DoS} & \\ \hline
			B & DoS & \autoref{sec:DoS} & host permission \\ \hline
			C & Download and open file & \autoref{sec:downloads} & downloads + downloads.open + downloads.shelf \\ \hline
			C & Exchange downloaded File & \autoref{sec:downloads} & downloads \\ \hline
			A & Steal cookies & \autoref{sec:stealCookies} & \\ \hline
			C & Steal cookies & \autoref{sec:stealCookies} & cookies + host permissions \\ \hline
			C & disable other extensions & \autoref{sec:disableOtherExtension} & management \\ \hline
			C & remove response headers & \autoref{sec:removeSecurityHeaders} & host permissions + webRequest + webRequestBlocking \\ \hline
		\end{tabular}
		\caption{Categorization of our components sorted by section.}
		\label{tab:componentCategorization}
	\end{table}
	
\clearpage
\section{Results}
	
	In this section, we present the results of our extension analysis. We extracted needed information for our analysis from the extensions' manifests. \autoref{tab:analysisSummaryContent} and \autoref{tab:analysisSummaryPermissions} show a summary of our analysis.
	
	Our analysis revealed that we can integrate all components from group A into 13 (86\%) and all components from group B into 12 (80\%) of the 15 analyzed extensions. This allows us to execute the uppermost part of our implementations, especially from the identification section where the general fingerprinting techniques (\autoref{sec:additionalFingerprintingData}) and reading out personal user information (\autoref{sec:personalUserInformation}) are the most efficient ways to identify the current user. Furthermore, we can integrate our component to store an unique identifier (\autoref{sec:storeIdentifier}) into 5 (33\%), our component that executes history sniffing (\autoref{sec:historySniffing}) into 3 (20\%), our component that remove security relevant HTTP response headers (\autoref{sec:removeSecurityHeaders}) into 7 (46\%), and our component that manipulates web requests using the \texttt{webRequests} module (\autoref{sec:manipulateWebRequests}) into 7 (46\%) of the analyzed extensions.
	
	Only 3 (20\%) analyzed extensions have enabled the use of \texttt{eval} and related functions in their background pages. This hampers our attack components that run in the extension's background process because we can not execute the fetched source code using \texttt{eval} in most extensions. But, we can use our second component for this task that fetches and executes a remote script using a script element in the extension's background page. To use this component, the extension needs a modified CSP with the targeted server's URL. We found out that 8 of the 15 analyzed extensions declare an additional origin in their CSP to remotely fetch scripts. 
	
	We found one extension that declares neither a content script, nor host permissions to all web pages, nor a modified CSP to fetch scripts remotely using script elements. The extension only requests host permissions for \texttt{*://*.google.com/}. This restricts our components from group A and B to only access web pages and remote server that match the declared host permission. Furthermore, the extension declares the API permissions \texttt{alarms}, \texttt{tabs}, and \texttt{webNavigation} which none of our components use. %TODO maybe prise this extension JAY JAY 

	2 of the 15 analyzed extensions declare the \texttt{downloads} permission but neither the \texttt{downloads.open} nor the \texttt{downloads.shelf} permissions. Therefore, we can not integrate our component that downloads and opens an arbitrary file in any of the analyzed extensions (\autoref{sec:downloads}). Our other two components that use the \textit{downloads} module to exchange a downloaded file can not hide their actions by disabling the browser bar that shows active downloads without the \texttt{downloads.shelf} permission. The user may notices that his genuine download was interrupted respectively removed and the extension has initiated another download itself. This reduces the success rate, but does not make the attack impossible.
	
	Only one of the analyzed extensions has declared the \texttt{management} permission. Thereby, we are able to integrate our attack component that disables another extension into it (\autoref{sec:disableOtherExtension}). The extension uses the permission to disable another extension that is known to be malicious. Because it declares the management permission as optional, the user has to explicitly allow the extension to disable other extensions. As soon as the user has granted the permission, our component can also use it to disable other extensions. % (Adblock even removes the permission if it has finished deactivating other extensions)

	Our analysis revealed, that we can integrate our component that collects cookies from the web pages that the user visits in 86\% of the analyzed extensions (\autoref{sec:stealCookies}). It uses only a simple content script and is limited to active web pages and cookies without the \texttt{httpOnly} flag enabled. To collect all cookies that the browser currently stores, we implemented another component that uses the cookies module. Our analysis revealed that 9 of 15 (60\%) of the analyzed extensions request the \texttt{cookies} permission and thereby allowing us to integrate our component into them. 
	
	\begin{table}[h]
		\centering
		\begin{tabular}{|l|l|} \hline
			Declares modified CSP and \texttt{unsafe\_eval} enabled? & 3/15 (20\%) \\ \hline
			Declares modified CSP and remotely loaded scripts enabled? & 8/15 (53\%) \\ \hline
			Uses a content script that matches \textbf{all} web pages? & 11/15 (73\%) \\ \hline
			Declares host permissions for \textbf{all} web pages? & 12/15 (80\%) \\ \hline
			Has host permissions or content script for \textbf{all} web pages? & 13/15 (86\%) \\ \hline
		\end{tabular}
		\caption{Summary and comparing of declared content of analyzed extensions.}
		\label{tab:analysisSummaryContent}
	\end{table}
	\begin{table}[h]
		\centering
		\begin{tabular}{|l|l|} \hline
			\textbf{API Permissions} &  \\ \hline
			\texttt{tabs} & 14/15 (93\%) \\
			\texttt{contextMenus} & 9/15 (60\%) \\
			\texttt{cookies} & 9/15 (60\%) \\
			\texttt{webNavigation} & 8/15 (53\%) \\
			\texttt{notifications} & 7/15 (46\%) \\
			\texttt{webRequest} & 7/15 (46\%) \\
			\texttt{webRequestBlocking} & 7/15 (46\%) \\
			\texttt{storage} & 5/15 (33\%) \\
			\texttt{idle} & 4/15 (26\%) \\
			\texttt{unlimitedStorage} & 4/15 (26\%) \\
			\texttt{history} & 3/15 (20\%) \\
			\texttt{nativeMessaging} & 3/15 (20\%) \\
			\texttt{activeTab} & 2/15 (13\%) \\
			\texttt{alarms} & 2/15 (13\%) \\
			\texttt{downloads} & 2/15 (13\%) \\
			\texttt{background} & 1/15 (6\%) \\
			\texttt{bookmarks} & 1/15 (6\%) \\
			\texttt{browsingData} & 1/15 (6\%) \\
			\texttt{clipboardWrite} & 1/15 (6\%) \\
			\texttt{clipboardRead} & 1/15 (6\%) \\
			\texttt{identity} & 1/15 (6\%) \\
			\texttt{management} & 1/15 (6\%) \\
			\texttt{proxy} & 1/15 (6\%) \\
			\texttt{privacy} & 1/15 (6\%) \\
			\texttt{system.cpu} & 1/15 (6\%) \\
			\hline
		\end{tabular}
		\caption{Summary and comparing of declared API permissions of analyzed extensions.}
		\label{tab:analysisSummaryPermissions}
	\end{table}
	

	%This extension provides several features to translate single words up to whole pages. It adds a context menu entry for the web page to translate highlighted text. A button in the browser's interface opens a pop-up in which the user can translate entered text or push a button to translate the whole page.
	
	%The extension uses the \texttt{activeTab} permission to inject a content script in the current web page if the user clicks the extension's context menu entry. However, the extension still injects the same content script in every web page rendering the privacy preserving functionality of the activeTab permission useless. This increases the scope of application of our implementation because we are able to integrate all components from group A. If the extension would inject the content script only with the activeTab functionality, our components were  

	%Hola provides a Virtual Private Network (VPN) as a free of charge extension. It routes the user's traffic through different countries to mask his true location. This allows to bypass regional restrictions on websites. A typical VPN network secures the web requests of its user's by routing the traffic to a few endpoints, masking the web request's origin. But Hola uses the devices of its unpaid customers to route traffic. It turns the user's computer into a VPN server and simultaneously to a VPN endpoint which means that the traffic of other users may exit through his Internet connection and take on his IP address. A Hola user's IP is therefore regularly exposed to the open Internet by traffic from other user's. The user himself has no possibility to control what content is loaded with his IP address as origin. The company makes money by providing the network to paying customers. Those are able to route their own traffic over the network to targeted endpoints. 
	
	%The paid functionality of Hola has strong similarities with a bot net which is used for denial of service or spmaming attacks. Actually, Hola recently received negative publicity as the owner of the web platform \textit{8chan} claimed that an attacker used the Hola network to perform a DDoS attack against his platform \cite{8chanHola}. Thereupon, researchers from the cyber security company \textit{Vectra}\footnote{Vectra Homepage: \url{http://www.vectranetworks.com/}} analyzed Hola's application and network. They discovered that Hola has - in addition to the public botnet-like functionality of routing huge amounts of targeted traffic - several features which may be used to perform further cyber attacks, such as download and execute any file while bypassing anti virus checking \cite{vectraHola}. 
	

